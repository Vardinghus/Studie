\chapter{Matematik}

\label{Label 1}

\section{yndlingsligninger}

\begin{align}
xˆ2 + yˆ2 - 4x + 8y + 11 = 0
\label{eq:ligning1}
\end{align}
\begin{align}
P(x) = \frac{x - a}{b \cdot a}
\end{align}


\begin{align}
x^3 + y^3 + z_1 + \SI{43}{m^3} - 2 \cdot 5
\end{align}

\begin{align}
\sum x_1 = x_hav + x_{s\textit{ø}} + \vec{x_fjord}
\end{align}

\section{Matematik i brødtekst}

Enheden for volumen er typisk \si{m^3}. Tyngdeaccelerationen er i Danmark \SI{9,82}{m/sˆ2}. Brug
af si-makroerne giver en pæn og konsistent præsentation af matematik. Til græske bogstaver eller specialtegn
bruges \$\$-konstruktionen, f.eks. $\alpha$, $\Rightarrow$ eller 20 $\decC$.




\begin{align}
\Phi = \rho \m c_p \m q_v \m \Delta T
\label{eq:flux}
\end{align}

Hvor:
\begin{table}[H]
\begin{tabular}{l|l}
$\Phi$ & Varmestrøm [\si{W}] \\
$\rho$ & Massefylde [\si{kg/mˆ3}] \\
$c_p$ & Varmefylde [\si{J/kgK}] \\
$\Delta T$ & Temperaturforskel [\si{K}]
\end{tabular}
\end{table}


