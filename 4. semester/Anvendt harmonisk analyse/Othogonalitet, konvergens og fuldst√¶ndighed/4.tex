\documentclass[12pt,a4paper,draft]{report}
\usepackage[utf8]{inputenc}
\usepackage{amsmath}
\usepackage{amsfonts}
\usepackage{amssymb}
\author{Frederik Appel Vardinghus-Nielsen}
\begin{document}
For $f$ og $g$ stykvis kontinuerte funktioner på $[a,b]$ definerer vi
\begin{align*}
\left<f,g\right>=\int_a^b\! f(x)\overline{g(x)}\,dx\\
||f||=\sqrt{\int_a^b \! |f(x)|^2\, dx}\\
\end{align*}
Man kan vise for $f$ og $g$, at
\begin{align*}
|\left<f,g\right>|\leq ||f|| \ ||g||\phantom{mm}\text{Cauchy-Schwartz}\\
||f+g||\leq ||f||+||g||\phantom{mm}\text{$\Delta$-uligheden}
\end{align*}
%Betragt funktionerne $\phi_n(x)=\frac{1}{\sqrt{2\pi}}\mathrm{e}^{inx}$ på $[-\pi,\pi]$.
%\begin{equation}\left<\phi_m,\phi_n\right>=\frac{1}{2\pi}\int_{-\pi}^{\pi}\! \mathrm{e}^{i(m-n)x\,dx=\begin{cases}0\phantom{mm},m\neq n\\1\phantom{mm},m=n\end{cases}\end{equation}

Fourierrækken for $f$ er
\begin{equation}
\sum_{-\infty}^{\infty}c_n\mathrm{e}^{inx}=\sum_{-\infty}^{\infty}\left<f,\phi_n\right>\phi_n
\end{equation}
For at sørge for, at Fourierrækker konvergerer skal man bestemme for hvilke funktioner dette gælder. Rummet bestående af alle stykvis kontinuerte funktioner er ikke fuldstændigt. Det fuldstændiggøres med
\begin{equation}
L^2(a,b)=\{f:\int_a^b\!|f(x)|^2\,dx<\infty\}
\end{equation}
\textbf{Ortonormal familie}\\
En familie af funktioner $\{f_k\}_{k=1}^{\infty}$ i $L^2(a,b)$ kaldes ortonormal hvis
\begin{equation}
\left<f_m,f_n\right>=\begin{cases}0\phantom{mm},m\neq n\\1\phantom{mm},m=n\end{cases}
\end{equation}
\textbf{Bessels ulighed}
\begin{equation}
\sum_{n=1}^{\infty}|\left<f,\phi_n\right>|^2\leq ||f(x)||^2
\end{equation}



\end{document}