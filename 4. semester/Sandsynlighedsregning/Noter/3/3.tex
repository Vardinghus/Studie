\documentclass[12pt,a4paper,final]{report}
\usepackage[utf8]{inputenc}
\usepackage{amsmath}
\usepackage{amsfonts}
\usepackage{amssymb}
\author{Frederik Appel Vardinghus-Nielsen}
\begin{document}
\noindent \textbf{Stochastic (random) variables}\\\\
\textbf{Definition}\\
A random variable $X$ is a function that takes its values from the sample space.
\begin{equation}
X:S\to\mathbb{R}
\end{equation}
For a set A we note
\begin{equation}
P(X\in A)=P(\{s<in S,X(s)\in A\})
\end{equation}
\textbf{Comulative distribution function}\\
Let $X$ be a random variable with cumulative distribution function $F$. Then $F$ verfies the following properties:
\begin{enumerate}
\item For $a,b\in \mathbb{R}$ with $a<b$\\
$F(b)=P(x\leq a)+P(a<X\leq b)=F(a)+P(a<x\leq b)$
\item $F(x)=1-P(X>x)$, for $x\in \mathbb{R}$
\item $F$ is \textit{cadlag}: right-continuous with limits from the left
\item $\lim_{x\to -\infty}F(x)=0$ and $\lim_{x\to \infty}F(x)=1$
\end{enumerate}
The cumulative distribution funtion (cdf) $F_x$ characterizes $X$ completely.\\\\
\textbf{Probability mass function}\\
Let $\{x_1,x_2,...\}$ be the values of the discrete random variable $X$.\\
The function $p(x_k)=P(X=x_k),\phantom{mm}k=1,2,...$ is called the probability mass function.\\\\
Let $X$ be a random variable with cdf $F$ and pdf $p$.\\
Then we have the following properties:
\begin{enumerate}
\item $F(x_k)=\sum$ Nåede ikke mere. Lårt.
\end{enumerate}
Står i bogen.\\\\
\textbf{Continuous variable}\\
If $F_x$ is continuously differentaible, $X$ is called a continuous random variable.\\
The function $f=F_x'$ is called the qualitatively density function of $X$.\\\\
Let $X$ be a continuous random variable with cdf $F$ and pdf $f$.\\
Then
\begin{enumerate}
\item $F(x)=\int_{-\infty}^{x} f(t) dt$
\item $f(x)=F_x'(x)$
\item $P(x\in B)=\int_B f(x)dt$
\end{enumerate}
$f$ characterizes $X$ completely.\\\\
\textbf{The uniform distribution}\\
$X$ is a unform distribution on $[a,b]$ for $a<b$ if $f(x)=\frac{1}{b-a}$ if $x\in [a,b]$.\\
We write $X\sim Unif([a,b])$\\\\
Let $X\sim Unif([0,1])$ and $Y=a+(b-a)X$.\\
Then $Y\sim Unif([a,b])$\\
\textit{Proof} For $y\in\mathbb{R}$\\
$F_Y(y)=P(Y\leq y)=P(a+(b-a)X\leq y)=P(X\leq\frac{y-a}{b-a})$\\
A $X$ in $Unif([0,1])$; we have
\begin{enumerate}
\item $F_X(x)=x$ if $f\in [0,1]$
\item $F_X(x)=0$ if $x\leq 0$
\item $F_X(x)=1$ if $x\geq 1$
\end{enumerate}
Hence
\begin{enumerate}
\item $F_Y(y)=\frac{y-a}{b-a}$ if $y\in [a,b]$
\item $F_X(y)=0$ if $y\leq a$
\item $F_Y(y)=1$ if $y\geq 1$
\end{enumerate}
Therefore $f_y(y)=F_Y'(y)=\frac{1}{b-a}$ if $y\in [a,b]$.\\\\
Let $X$ be a random variable, $a,c\in\mathbb{R}$\\
Then
\end{document}