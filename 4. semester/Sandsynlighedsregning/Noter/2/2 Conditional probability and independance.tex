\documentclass[12pt,a4paper,draft]{report}
\usepackage[utf8]{inputenc}
\usepackage{amsmath}
\usepackage{amsfonts}
\usepackage{amssymb}
\author{Frederik Appel Vardinghus-Nielsen}
\begin{document}
\paragraph{Conditional probability}
What is the probability of $A$ given $B$?
\begin{equation}
\# A \text{ knowing } B=\frac{\# A\cap B}{\# B}
\end{equation}
\paragraph{Definition (Conditional probability)}
Let $B$ be an event with $P(B)>0$.\\
For ant event $A\in S$
\begin{equation}
P(A/B)=\frac{P(A\cap B)}{P(B)}
\end{equation}
Putting in the same way as the axioms of probability:
\paragraph{Proposition} For any event $B$ with $P(B)>0$, $P(\cdot / B)$ is a probability
\begin{itemize}
\item For $A\in S$, $0\leq P(A/B)\leq 1$
\item $P(S/B)=1$
\item $P(\cup_{k=1}^{+\infty} A_k/B)=\sum_{k=1}^{+\infty}P(A_k/B)$ for $A_i\cap A_j=Ø$ for $i<j$.
\end{itemize}
\paragraph{Example} Roll a die and observe the number
\begin{align}
A=\{\text{an odd number}\}\\
B=\{\text{at least 4}\}
\end{align}
This gives
\begin{equation}
P(A/B)=\frac{P(A\cap B)}{P(B)}=\frac{\frac{1}{6}}{\frac{3}{6}}=\frac{1}{3}
\end{equation}
\paragraph{Properties}
\begin{align}
P(A^c/B)&=1-P(A/B)\\
P(A\c/B&=P(A/B)-P(A\cap c/B)\\
P(A\cup c/B)&=P(A/B)+P(c/B)-P(A\cap c/B)\\
\text{If }A\in c, \ P(A/B)&\leq P(c/B)
\end{align}
\textbf{Independence}\\
Let $A$ and $B$ be two events\\
$A$ is independent of $B$ if $P(A\cap B)=P(A)P(B)$\\
Remark: Assume $P(B)>0$.\\
$P(A/B)=\frac{P(A\cap B)}{P(b)}=\frac{P(A)P(B)}{P(B)}=P(A)$\\\\
If $A$ is independant of $B$ then $B$ is independant of $A$.\\\\
\textbf{Example}\\
Choose a card from a deck (52 cards)
\begin{align}
A&=\{ace\}\\
B&=\{heart\}
\end{align}
Are $A$ and $B$ independant?\\
The probability of drawing ace of hearts:
\begin{equation}
P(A\cap B)=1/52
\end{equation}
The probability of $A$:
\begin{equation}
P(A)=4/52
\end{equation}
The probability of $B$:
\begin{equation}
P(B)=13/52=1/4
\end{equation}
We have $P(A\cap B)=P(A)P(B)$ so $A$ and $B$ are independant.\\\\
\textbf{Definition}\\
Three events $A$, $B$ and $C$ are said to be mutually independant if
\begin{itemize}
\setlength\itemsep{0em}
\item[a)] They are pairwise independant
\item[b)] $P(A\cap B\cap C)=P(A)P(B)P(C)$
\end{itemize}
\textbf{Definition}\\
For $n=1$ and $A_1,\hdots ,A_n\in S$ the events $A_1,\hdots ,A_n$ are mutually independant if 
\begin{itemize}
\item[a)] they are pairwise independant
\item[b)] any combination of $A$'s is mutually independant
\end{itemize}
\textbf{Proposition (Law of Total Probability)}\\
For any events $B_1,B_2,\hdots$ such that 
\begin{itemize}
\item $P(B_i)>0$ for $i=1,2,\hdots$
\item $B_i\cap B_j=Ø$ for $i\neq j$ and $i,j=1,2,\hdots$
\item $\cup_{i=1}^{+\infty}B_i=S$
\end{itemize}
$B_1,B_2,\hdots$ is called a partition of $S$. And for any event $A$, $P(A)=\sum_{i=1}^{+\infty}P(A/B_i)P(B_i)$
\end{document}
