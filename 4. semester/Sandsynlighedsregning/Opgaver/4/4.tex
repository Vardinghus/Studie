\documentclass[12pt,a4paper,draft]{report}
\usepackage[utf8]{inputenc}
\usepackage{amsmath}
\usepackage{amsfonts}
\usepackage{amssymb}
\author{Frederik Appel Vardinghus-Nielsen}
\begin{document}
\noindent \textbf{Opgave 28}\\
I a street bet in roulette you bet on three numbers. If any of these come up, you win 11 times your wager, otherwise you lose your wager. Let $X$ be your gain if you bet one dollar on a street bet. Find the mean and variance of $X$.\\\\
Der vindes på tre forskellige tal og tabes på 35 forskellige.
\begin{equation}
E[X]=11\frac{3}{38}+(-1)\frac{35}{38}=-\frac{2}{38}
\end{equation}
Varians beregnes ved nedenstående formel.
\begin{equation}
Var[X]=E[X^2]-E[X]^2=11^2\frac{3}{38}+\left((-1)\frac{35}{38}\right)^2
\end{equation}
\textbf{Opgave 30}\\
The game of chuck-a-luck is played with three dice, rolled independently. You
bet one dollar on one of the numbers 1 through 6 and if exactly k of the dice show
your number, you win k dollars k = 1, 2, 3 (and keep your wagered dollar). If
no die shows your number, you lose your wagered dollar. What is your expected
loss?\\\\
\begin{equation}
E[X]=(-1)\left(\frac{5}{6}\right)^3+1\frac{5^2\cdot3}{6^3}+2\frac{3\cdot5}{6^3}+3\left(\frac{1}{6}\right)=-0.078703\text{ cent}
\end{equation}
\textbf{Opgave 35}\\
A stick measuring one yard in length is broken into two pieces at random.
Compute the expected length of the longest piece.\\\\
Først bestemmes en funktion $g(x)$.
\begin{equation}
g(x)=\max(x,1-x)
\end{equation}
Derefter bruges relationen for gennemsnit af $g(x)$:
\begin{align}
E[g(x)]&=\int_0^1\!g(x)f_X(x)\,dx\\
&=\int_0^1\!g(x)\,dx\\
&=\int_0^{0.5}\!1-x\,d+\int_{0.5}^1\!x\,dx\\
&=\frac{3}{4}
\end{align}
\textbf{Opgave 38}\\
The random variable $X$ has pdf $f(x)=3x^2,0\geq x\geq 1$. \textbf{(a)} Compute $E[X]$ and
$Var[X]$. \textbf{(b)} Let $Y=\sqrt{X}$ and compute $E[Y]$ and $Var[X]$.\\\\
\textbf{(a)}
\begin{align}
E[X]&=\int_0^1\!xf(x)\,dx\\
&=\int_0^1\!3x^3\,dx\\
&=\left[\frac{3}{4}x^4\right]_0^1\\
&=\frac{3}{4}
\end{align}
\textbf{(b)}
\begin{align}
Var[X]&=E[(X-\frac{3}{4})^2]\\
&=E[X^2+\frac{9}{16}-\frac{3}{2}X]\\
&=E[X^2]-\frac{3}{2}E[X]+\frac{9}{16}\\
&=\frac{3}{5}-\frac{9}{8}+\frac{9}{16}\\
&=\frac{3}{80}=0.0375
\end{align}
\textbf{Opgave 39}\\
Let $X\geq 0$ be continuous. Show that $E[X]=\int_0^{\infty}\!P(X>x)\,dx$ and $E[X^2]=2\int_0^{\infty}\!xP(X>x)\,dx$.\\\\
Der startes med definitionen
\begin{align}
E[X]&=\int_0^{\infty}\!xf(x)\,dx\\
&=\int_0^{\infty}\!\int_0^x\!f(t)\,dt\,dx\\
&=\int_0^{\infty}\!\int_t^{\infty}\!f(x)\,dx\,dt\\
&=\int_0^{\infty}\!P(x\in [t,\infty)\,dt\\
&=\int_0^{\infty}\!P(X>x)\,dx
\end{align}





\end{document}