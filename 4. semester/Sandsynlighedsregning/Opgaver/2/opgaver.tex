\documentclass[12pt,a4paper,draft]{report}
\usepackage[utf8]{inputenc}
\usepackage{amsmath}
\usepackage{amsfonts}
\usepackage{amssymb}
\author{Frederik Appel Vardinghus-Nielsen}
\begin{document}
\paragraph{Opgave 38} Let $A$ and $B$  be disjoint  events. Show that
\begin{equation}
P(A/A\cup B)=\frac{P(A)}{P(A)+P(B)}
\end{equation}
Definition 1.4 bruges. Så fås
\begin{equation}
P(A/A\cup B)=\frac{P(A\cap (A\cup B))}{P(A\cup B)}
\end{equation}
Den distributive lov bruges og regel om foreningsmængder, som ingen skæring har, bruges, og der fås
\begin{equation}
\frac{P(A\cap (A\cup B))}{P(A\cup B)}=\frac{P(A\cap A)\cup P(A\cap B)}{P(A\cup B)}=\frac{P(A)}{P(A)+P(B)}
\end{equation}
\paragraph{Opgave 43}
Show that both $\emptyset$ and the sample space $S$ are independent  of any event. Explain intuitively.\\\\
Brug definitionen for uafhængighed og sæt $S$ ind.
\begin{align}
P(A\cap B)&=P(A)P(B)\\
P(S\cap B)&=P(S)P(B)
\end{align}
Da $P(S\cap B)=P(B)$ og $P(S)=1$ fås
\begin{equation}
P(B)=P(B)
\end{equation}
og udsagnet er altså sandt.\\
Det samme gøres med den tomme mængde. Da $P(\emptyset)=0$ og $P(\emptyset\cap B)=0$ fås
\begin{align}
P(\emptyset \cap B)&=P(S)P(B)\\
0&=0
\end{align}
\paragraph{Opgave 46}
A fair coin is flipped twice. Explain the difference between  the following: \textbf{(a)} the probability  that both flips  give heads, and \textbf{(b)} the conditional probability that the seecond flip gives heads given that the first flip gave heads.\\\\
Se, at 
\begin{align}
S&=\{HH,HT,TH,TT\}\\
A&=\{HH\}\\
B&=\{HT,HH\}
\end{align}
Det ses dermed, at
\begin{align}
P(A)&=\frac{\# A}{\# S}=\frac{1}{4}\\
P(B)&=\frac{\# B}{\# S}=\frac{1}{2}
\end{align}
Formlen for betinget sandsynlighed er dermed
\begin{align}
P(A/B)&=\frac{P(A\cap B)}{P(B)}\\
&=\frac{P(A)}{P(B)}\\
&=\frac{1}{2}
\end{align}
Det ses, at de to sandsynligheder er forskellige. Rent intuitivt sker dette fordi der i \textbf{(b)} er elimineret to af mulighederne i $S$ fra starten.
\paragraph{Opgave 50}
You roll a dice twice  and record the largest number. \textbf{(a)} Given that the first roll gives 1, what is the conditional probability that the largest number is 3? \textbf{(b)} Given that the first roll gives 3, what is the conditional probability that the largest number is 3?\\\\
Lad $S=\{(1,1),(1,2),\hdots,(1,6)\}$ og $A=\{(1,3)\}$.
Da haves
\begin{equation}
P(A)=\frac{\# A}{\# S}=\frac{1}{6}
\end{equation}
og
\begin{equation}
P(A/S)=\frac{P(A\cap S)}{P(S)}=\frac{P(A)}{1}=P(A)=\frac{1}{6}
\end{equation}
Lad nu $S=\{(3,1),(3,2),\hdots,(3,6)\}$ og $A=\{(3,1),(3,2),(3,3)\}$. Da haves
\begin{equation}
P(A)=\frac{\# A}{\# S}=\frac{1}{2}
\end{equation}
og
\begin{equation}
P(A/S)=\frac{P(A\cap S)}{P(S)}=\frac{P(A)}{1}=P(A)=\frac{1}{2}
\end{equation}
\paragraph{Opgave 73}
You roll a die and flip a fair coin a number of times  determined by the number on the die. What is the probability that you get no heads?\\\\

\paragraph{Opgave 87}

\end{document}