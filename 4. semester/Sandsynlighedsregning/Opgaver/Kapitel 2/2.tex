\documentclass[12pt,a4paper,draft]{report}
\usepackage[utf8]{inputenc}
\usepackage{amsmath}
\usepackage{amsfonts}
\usepackage{amssymb}
\author{Frederik Appel Vardinghus-Nielsen}
\begin{document}
\noindent\textbf{Opgave 53}\\
Let $X$ be the number of 6s when a die is rolled six times, and let $Y$ be the number of 6s when a die is rolled 12 times. Find \textbf{(a)} $E[X]$ and $E[Y]$ and \textbf{(b)} $P(X\geq E[X])$ and $P(Y\geq E[Y])$.\\\\
\textbf{(a)} Forventede værdi beregnes ved\
\begin{align*}
E[X]&=\sum_{k=\infty}^{\infty}x_kP(X=x_k)=\sum_{\infty}^{\infty}x_kp(x_k)\\
&=\sum_{k=0}^6\begin{pmatrix}
6\\k
\end{pmatrix}
\left(\frac{1}{6}\right)^k\left(\frac{5}{6}\right)^{6-k}k\\
&=1
\end{align*}
På samme måde med $E[Y]$.
\begin{align*}
E[Y]=\sum_{k=0}^{12}\begin{pmatrix}
12\\k
\end{pmatrix}
\left(\frac{1}{6}\right)^k\left(\frac{5}{6}\right)^{12-k}k\\
&=2
\end{align*}
\textbf{(b)} Sandsynlighederne for at få 1 eller flere 6'ere summeres.
\begin{equation}
P(X\geq E[X])=\sum_{k=1}^6\begin{pmatrix}
6\\k
\end{pmatrix}
\left(\frac{1}{6}\right)^k\left(\frac{5}{6}\right)^{6-k}=0.665
\end{equation}
Det samme gøres for $Y$.
\begin{equation}
P(X\geq E[X])=\sum_{k=2}^{12}\begin{pmatrix}
12\\k
\end{pmatrix}
\left(\frac{1}{6}\right)^k\left(\frac{5}{6}\right)^{12-k}=0.6187
\end{equation}
\textbf{Opgave 66}\\
\textbf{(a)} Flip a coin 10 times and let $X$ be the number of heads. Compute $P(X\leq 1)$
exactly and with the Poisson approximation. \textbf{(b)} Now instead flip four coins 10
times and let $X$ be the number of times you get four heads. Compute $P(X\geq 1)$
exactly and with the Poisson approximation. \textbf{(c)} Compare \textbf{(a)} and \textbf{(b)}. Where does the approximation work best and why?\\\\
\textbf{(a)} Sandsynligheden beregnes eksakt.
\begin{align*}
\sum_{k=0}^1\begin{pmatrix}
10\\
k
\end{pmatrix}
\left(\frac{1}{2}\right)^k\left(\frac{1}{2}\right)^{10-k}=0.0107
\end{align*}
Parameteret $\lambda$ er gennemsnittet af krone - altså 5. Approksimationen er dermed.
\begin{equation}
P(X\leq 1)=\sum_{k=0}^1\frac{5^k}{k!}\mathrm{e}^{-5}=0.0404
\end{equation}
\textbf{(b)} Beregn hvor tit der fås 4 kroner ved 10 kast med 4 mønter.











\end{document}