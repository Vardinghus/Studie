\documentclass[12pt,a4paper,draft]{report}
\usepackage[utf8]{inputenc}
\usepackage{amsmath}
\usepackage{amsfonts}
\usepackage{amssymb}
\author{Frederik Appel Vardinghus-Nielsen}
\begin{document}
\noindent\textbf{Opgave 9}\\
Markov chains are named for Russian mathematician A. A. Markov, who in the early twentieth century examined the sequence of vowels and consonants in the 1833 poem \textit{Eugene Onegin} by Alexander Pushkin. He empirically verified the Markov property and found that a vowel was followed by a consonant 87\% of the time and a consonant was followed by a vowel 66\% of the time. \textbf{(a)} Give the transition graph and the transition matrix. \textbf{(b)} If the first letter is a vowel, what is the probability that the third is also a vowel? \textbf{(c)} What are the proportions of vowels and consonants in the text?\\\\
\textbf{(a)} Transitionsmatricen ser således ud
\begin{equation}
P=\begin{bmatrix}
0.34 & 0.66\\
0.87 & 0.13
\end{bmatrix}
\end{equation}
\textbf{(b)} Hvis det første bogstav er en vokal, så aflæses sandsynligheden for at tredje bogstav er en vokal fra $P^2$:
\begin{equation}
P^2=\begin{bmatrix}
0.6898 & 0.3102\\
0.4089 & 0.5911
\end{bmatrix}
\end{equation}
Sandsyndligheden for at tredje bogstav er en vokal er 0.5911.
\textbf{(c)} Mængden af vokaler og konsonanter findes ved
\begin{equation}
\frac{1}{q+p}\begin{bmatrix}
q & p\\
q & p
\end{bmatrix}=\begin{bmatrix}
0.57 & 0.43\\
0.57 & 0.43
\end{bmatrix}
\end{equation}
\textbf{Opgave 15}\\
Consider the success run chain in Example 8.16. Suppose that the chain has been running for a while and is in state 10. \textbf{(a)} What is the expected number
of steps until the chain is back at state 10? \textbf{(b)} What is the expected number of times the chain visits state 9 before it is back at 10?\\\\
\textbf{(a)} Brug proposition 8.5, som siger
\begin{equation}
E_i[\tau_i]=\frac{1}{\pi_i}
\end{equation}
Fra eksempel 8.16 haves
\begin{equation}
\pi_i=\begin{bmatrix}
1/2 & 1/2	& 0		& 0		& \cdots\\
1/2 & 0		& 1/2	& 0		& \cdots\\
\vdots & \vdots & \vdots & \ddots &\ddots
\end{bmatrix}=\frac{1}{2^{i+1}}
\end{equation}
Altså
\begin{equation}
E_{10}[\tau_{10}]=2^{10+1}=2048
\end{equation}
\textbf{(b)} Brug samme proposition 8.5. Der haves, at
\begin{equation}
E_{10}[N_9]=\frac{\pi_9}{\pi_{10}}=\frac{2^{11}}{2^{10}}=2
\end{equation}
\textbf{Opgave 16}\\
Let $g:[0,1]\to R$ be a function whose integral $I=\int^1_0\!g(x)\,dx$ is impossible to compute explicitly. How can you approximate $I$ by simulation of standard uniforms $U_1,U_2,\ldots$?\\\\
Brug opgave 3 fra kursusgang 14, hvor en function approksimeres ved en sum af uniforme distributioner.
\begin{equation}
I=\int_0^1\!g(x)\,dx\approx \lim_{n\to\infty}\sum_{k=1}^ng(U_k)/n
\end{equation}


\end{document}
















