\documentclass[12pt,a4paper]{report}
\usepackage[utf8]{inputenc}
\usepackage{amsmath}
\usepackage{amsfonts}
\usepackage{amssymb}
\author{Frederik Appel Vardinghus-Nielsen}
\begin{document}
\noindent{\Huge 8 Markov chains}\\\\
\begin{itemize}
\setlength\itemsep{0em}
\item Define Markov chains
\item Make example of on/off
\item Communication
\item Irreducibility
\item Recurrence
\item Stationary distribution
\item Convergence
\end{itemize}
\textbf{Definition 8.1. (Markov chain).} Let $X_0,X_1,\ldots$ be a sequence of discrete random variables, taking values in som set $S$ and that are such that
\begin{equation}
P(X_{n+1}=j|X_0=i_0,\ldots,X_{n-1}=i_{n-1},X_n=i)=P(X_{n+1}=j|X_n=i)
\end{equation}
for all $i,j,i_0,\ldots i_{n-1}$ in $S$ and all $n$. The sequence $\{X_n\}$ is the called a \textit{Markov chain}.\\\\
\begin{equation}
p_{ij}=P(X_{n+1}|X_n=j)
\end{equation}
\textbf{On/Off example} Probability $p$ of switching on and $q$ of switching of.
\begin{equation}
P=\begin{pmatrix}
1-p&p\\
q&1-q
\end{pmatrix}
\end{equation}
Limit distribution
\begin{equation}
\lim_{n\to\infty}P^{(n)}=\frac{1}{p+q}\begin{pmatrix}q&p\\q&p\end{pmatrix}
\end{equation}
\textbf{Definition 8.2. (Communication).} If $p_{ij}^{(n)}>0$ for som $n$, we say that state $j$ is accessible from state $i$ written $i\to j$. If both ways, they communicate.\\\\
\textbf{Definition 8.3. (Irreducibility).} If all states in $S$ communicate with each other, the Markov chain is said to be \textit{irreducible}.\\\\
\textbf{Definition 8.4. (Recurrence).} Consider a state $i\in S$ and $\tau_i$ the number of steps to first visit $i$:
\begin{equation}
\tau_i=\min\{n\geq1:X_n=i\}
\end{equation}
\begin{itemize}
\setlength\itemsep{0em}
\item[] $\tau_i=\infty$ never visited
\item[] Recurrent if $P_i(\tau_i<\infty)=1$
\item[] Transcient if $P_i(\tau<\infty)<1$
\end{itemize}
\textbf{Definition 8.8 (Periodicity).} The \textit{period} of state $i$ is defined as
\begin{equation}
d(i)=gcd\{n\geq1:p_{ij}^{(n)}>0\}
\end{equation}
If $d(i)=1$, $i$ is said to be aperiodic; otherwise periodic.\\\\
\textbf{Definition 8.5 (Stationary distribution).} Let $P$ be the transition matrix of a Markov chain with state space $S$. A probability distribution $\mathbf{\pi}=(\pi_1,\pi_2,\ldots)$ on $S$ satisfying
\begin{equation}
\mathbf{\pi}P=\mathbf{\pi}
\end{equation}
is called a stationary distribution of the chain.\\\\
\textbf{Theorem 8.1 (Convergence).} Consider an irreducible, positive recurrent, and aperiodic Markov chain with stationary distribution $\mathbf{\pi}$ and $n$-step transition matrix $p_{ij}^{(n)}$. Then
\begin{equation}
p_{ij}^{(n)}\to\pi_j\text{ as }n\to\infty
\end{equation}
for all $i,j\in S$.









\end{document}