\documentclass[12pt,a4paper,draft]{report}
\usepackage[utf8]{inputenc}
\usepackage{amsmath}
\usepackage{amsfonts}
\usepackage{amssymb}
\author{Frederik Appel Vardinghus-Nielsen}
\begin{document}
\noindent{\Huge 4 Two random variables -- independence}\\\\
Define
\begin{itemize}
\setlength\itemsep{0em}
\item Discrete random vector
\item Joint pmf
\end{itemize}
Prove
\begin{itemize}
\setlength\itemsep{0em}
\item Marginal pmfs
\item Independence if and only if
\end{itemize}
\textbf{Definition 3.3. (Discrete random vector)} If $X$ and $Y$ are discrete random variables, then $(X,Y)$ is called a \textit{discrete random vector}.\\\\
\textbf{Definition 3.4. (Joint pmf)} If $(X,Y)$ is discrete with range $\{(x_j,y_k):j,k=1,2,\ldots\}$, the function
\begin{equation}
p(x_j,y_k)=P(X=x_j,Y=y_k)
\end{equation}
is called the \textit{joint pmf} of $(X,Y)$.\\\\
\textbf{Proposition 3.2 (Marginal pmfs)} If $(X,Y)$ has joint pmf $p$, then the marginal pmfs of $X$ and $Y$ are
\begin{align*}
p_X(x_j)&=\sum_{k=1}^{\infty}p(x_j,y_k),\phantom{mm}j=1,2,\ldots\\
p_Y(y_k)&=\sum_{j=1}^{\infty}p(x_j,y_k),\phantom{mm}k=1,2,\ldots
\end{align*}
\textbf{Bevis}\\
\begin{align*}
p_X(x_j)=P(X=x_j)&=P\left(\{X=x_j\}\cap\{\cup_{k=1}^{\infty}Y=y_k\}\right)\\
&=P\left(\cup_{k=1}^{\infty}\{X=x_j,Y=y_k\}\right)\\
&=\sum_{k=1}^{\infty}P(X=x_j,Y=y_k)=\sum_{k=1}^{\infty}p(x_j,y_k)
\end{align*}
\textbf{Definition 3.8. (Independence)} The random variables $X$ and $Y$ are said to be $independent$ if
\begin{equation}
P(X\in A,Y\in B)=P(X\in A)P(Y\in B)
\end{equation}
for alle $A,B\subseteq\mathbb{R}$.\\\\
\textbf{Proposition 3.9. (Independence if and only if)} Suppose that $(X,Y)$ is discrete with joint pmf $p$. Then $X$ and $Y$ are independent if and only if
\begin{equation}
p(x,y)=p_X(x)p_Y(y)
\end{equation}
for all $x,y\in\mathbb{R}$.\\\\
\textbf{Bevis}\\
Suppose $X$ and $Y$ independent. Let $A=\{x\}$ and $B=\{y\}$ so that
\begin{equation}
p(x,y)=P(X=y,Y=y)=P(X=x)P(Y=y)=p_X(x)p_Y(y)
\end{equation}
Conversely, suppose $p(x,y)=p_X(x)p_Y(y)$ and take  the subsets $A$ and $B$ of $\mathbb{R}$.
\begin{align*}
P(X\in A,Y\in B)&=\sum_{x\in A}\sum_{y\in B}p(x,y)=\sum_{x\in A}p_X(x)\sum_{y\in B}p_Y(y)\\
&=P(X\in A)P(Y\in B)
\end{align*}
Both ``if'' and ``only if'' have been shown.


















\end{document}