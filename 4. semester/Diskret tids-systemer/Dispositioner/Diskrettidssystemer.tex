	\documentclass[12pt,a4paper]{report}
\usepackage[utf8]{inputenc}
\usepackage{amsmath}
\usepackage{amsfonts}
\usepackage{amssymb}
\author{Frederik Appel Vardinghus-Nielsen}
\begin{document}
\noindent{\Huge 1 Impulsrespons}\\\\
\textbf{LTI system}
\begin{align*}
y[n]&=T\left(\sum_{k=-\infty}^{\infty}x[n]\delta[n-k]\right)\\
&=\sum_{k=-\infty}^{\infty}x[n]T\left(\delta[n-k]\right)\\
&=\sum_{k=-\infty}^{\infty}x[n]h[n-k]
\end{align*}
\textbf{Lineære differensligninger med konstante koefficienter}
\begin{align*}
\sum_{k=0}^Na_ky[n-k]=\sum_{m=0}^Nb_mx[n-m]
\end{align*}
LTI er BIBO-stabilt hvis og kun hvis impulsresponsen ligger i $\ell^1$:
\begin{align*}
\sum_{k=-\infty}^{\infty}|h[n]|<\infty
\end{align*}
\textbf{Foldning og multiplikation i tid og frekvens}\\\\
\textbf{Digitale filtre}
\clearpage
\noindent{\Huge 2 Overføringsfunktioner}\\\\
Brug impulseresponsen som indledning.
\begin{align*}
y[n]&=x[n]*h[n]\\
Y(z)&=X(z)H(z)\\
H(z)&=\frac{Y(z)}{X(z)}
\end{align*}
\clearpage
\noindent{\Huge 3 Fouriertransformationen}\\\\
\begin{align*}
X[k]&=\sum_{n=0}^{N-1}x[n]e^{j\omega kn/N}\\
x[n]&=\sum_{k=0}^{N-1}X[k]e^{-j\omega kn/N}
\end{align*}
Sekvens, som er samples af DTFT.\\\\
\textbf{Frekvensanalyse}\\\\
\textbf{FIR-filterdesign}
\clearpage
\noindent{\Huge 4 Z-transformationen}
\begin{equation}
X(z)=\sum_{n=-\infty}^{\infty}x[n]z^{-1}
\end{equation}
Det diskrete modstykke til Laplace:
\begin{equation}
F(s)=\int_0^{\infty}\!f(t)e^{-st}\,dt
\end{equation}
\textbf{Region of convergence} Z-transformationen konvergerer for flere sekvenser end Fouriertransformationen.
\begin{equation}
|X(re^{j\omega})|=\sum_{n=-\infty}^{\infty}|x[n]r^{-n}|<\infty
\end{equation}
Hvis $r=1$ svarer ovenstående til Fouriertransformationen. Konvergensen afhænger af $Z$:
\begin{equation}
\sum_{n=-\infty}^{\infty}|x[n]||z|^{-n}<\infty
\end{equation}
\textbf{Nuller og poler}
\clearpage
\noindent{\Huge 5 Nyquist's samplesætning}\\\\
\textbf{Nyquist-Shannons samplsætning}: Lad $x_c(t)$ være et båndbegrænset signal, hvor
\begin{equation}
X_c(j\omega)=0\text{ for }|\Omega|\geq\Omega_N.
\end{equation}
Da er $x_c(t)$ unikt specificeret ved dets samples $x[n]=x_c(nT)\, ,n=0,\pm1,\pm2,\ldots$, hvis
\begin{equation}
\Omega_s=\frac{2\pi}{T}\geq 2\Omega_N
\end{equation}
således
\begin{equation}
x_c(t)=\sum_{n=-\infty}^{\infty}x_c(nT)\mathrm{sinc}(\omega_Nt-n\pi)
\end{equation}
\textbf{Fouriertransformation af $\delta$ og impulstog}
\begin{align*}
\mathcal{F}\{\delta(t)\}(\omega)&=1\\
\mathcal{F}\{\sum_{n=-\infty}^{\infty}\delta(t-nT)\}(\omega)&=\frac{2\pi}{T}\sum_{n=-\infty}^{\infty}\delta(\omega-n\omega_s)
\end{align*}
hvor $\omega_s=2\pi f_s$.\\\\
\textbf{Aliasering}\\\\
\textbf{Plancherels sætning} om energibevarelse
\clearpage
\noindent {\Huge 6 IIR-filtre, impulsinvariantmetoden og den bilineære transformation}\\\\
\textbf{Tjek semesterprojekt}\\\\
\textbf{IIR-filter}\\
Rekursiv algoritme, med uendelig impulsrespons -- fortsætter for evigt. Designes ud fra specifikationer omkring stop-, pas- og transitionsbånd.\\\\
\textbf{Butterworth}\\
\begin{equation}
|H(j\omega)|^2=\frac{1}{1+(j\omega/j\omega_c)^{2N}}
\end{equation}
\textbf{Impulsinvariantmetoden}\\
Et kontinuert filter samples:
\begin{equation}
h[n]=T_dh_c(nT_d)
\end{equation}
hvor $T_d$ er samplingsperioden. Frekvensresponsen for det samplede filter er
\begin{equation}
H(e^{j\omega})=\sum_{k=-\infty}^{\infty}H_c\left(j\frac{\omega}{T_d}+j\frac{2\pi}{T_d}k\right)
\end{equation}
Kontinuerte filtre er ikke båndbegrænsede, så der sker aliasering.\\\\
\textbf{Den bilineære transformation}\\
Motivation: analoge filtre er ikke båndbegrænsede. Den bilineære konverterer den reelle tallinje til enhedscirklen. Derfor sker forvrængning.
\begin{align*}
z&=e^{j\omega}\\
\omega&=T_d\Omega\\
s&=j\Omega\\
s&\approx\frac{2}{T_d}\frac{z-1}{z+1}\\
z&=\frac{1+\frac{T_d}{2}\sigma+j\frac{T_d}{2}\Omega}{1-\frac{T_d}{2}\sigma-j\frac{T-d}{2}\Omega}
\end{align*}
\textbf{Generelt for IIR-filtre}
\begin{itemize}
\item Opstil amplituderespons $|H_c(s)|$ for filter ud fra specifikationer
\item Find poler og brug dem i venstre halvplan til at opstille ny $|H_c(s)|$
\item Partialbrøksopspaltning
\item Invers Laplacetransformation
\item Sampling af impulsrespons
\item Z-transformation
\end{itemize}
\clearpage
\noindent{\Huge 7 Lineær fase, FIR-filtre og vinduesmetoden}\\\\
\textbf{Nødvendig betingelse for lineær fase}
\begin{equation}
\sum_{n=-\infty}^{\infty}h[n]\sin(\beta+(n-\alpha)\omega)=0
\end{equation}
Værdier somtilfredsstiller denne betingelse:
\begin{itemize}
\item $\beta=\begin{cases}0\\\pi\end{cases}$
\item $\alpha=\frac{M}{2}\,\Rightarrow\,2\alpha=M$
\item $h[n]=h[2\alpha-n]=h[M-n]$
\end{itemize}
\textbf{FIR-filtres poler}
\begin{align*}
H(z)&=\frac{Y(z)}{X(z)}=\frac{\sum_{i=0}^Mb_iz^{-i}}{1-\sum_{j=1}^Na_jz^{-j}}\\
&=\sum_{i=0}^Mb_iz^{-i}\bigg|_{a_j=0,\,j=1,\ldots,N}\\
&=\frac{\sum_{i=0}^{M}b_iz^{M-i}}{z^M}
\end{align*}

\clearpage
\noindent{\Huge 8 Geometrisk baseret analyse af amplitude- og fase-respons}\\\\ Se slides 20-25 i lektion 9.\\\\
Hvis overføringsfunktionen $H(z)$ haves, så kan der laves geometrisk analyse af denne.
\begin{equation}
H(z)=\frac{\sum_{k=0}^Mb_kz^-k}{\sum_{k=0}^Na_kz^{-k}}
\end{equation}
\begin{itemize}
\item $M=N$: forlæng $H(z)$ med $z^M$
\begin{equation}
H(z)=\frac{\sum_{k=0}^Mb_kz^{M-k}}{\sum_{k=0}^Mz^{M-k}}
\end{equation}
\item $M>N$: forlæng med $z^M$
\begin{equation}
H(z)=\frac{\sum_{k=0}^Mb_kx^{M-k}}{\sum_{k=0}^Ma_k^{M-k}}
\end{equation}
\item $M<N$: forlæng med $z^N$
\begin{equation}
H(z)=\frac{z^{N-M}\sum_{k=0}^Mb_Kz^{M-k}}{\sum_{k=0}^Na_kz^{N-k}}
\end{equation}
Faktorisér $H(z)$
\begin{equation}
H(z)=\frac{b_0}{a_0}\frac{\prod_{k=1}^L(z-c_k)}{\prod_{k=1}^L(z-d_k)},\phantom{mm}L=\max\{N,M\}
\end{equation}
\textbf{Amplituderespons}
\begin{equation}
|H(e^{j\omega})|=\left|\frac{b_0}{a_0}\right|\frac{\prod_{k=1}^L\left|e^{j\omega}-c_k\right|}{\prod_{k=1}^L\left|e^{j\omega}-d_k\right|}
\end{equation}
\textbf{Faserespons}
\begin{equation}
\arg\{H(e^{j\omega}\}=\arg\{\frac{b_0}{a_0}\}+\sum_{k=1}^L\arg\{e^{j\omega}-c_k\}-\sum_{k=1}^L\arg\{e^{j\omega}-d_k\}
\end{equation}
\end{itemize}
\clearpage
\noindent{\Huge 9 Frekvenstransformation af IIR-filtre}\\\\
Ud fra et prototype IIR-lavpasfilter transformeres dette om til en anden type. Generelt:
\begin{equation}
Z^{-1}=G(z^{-1})=\pm\prod_{k=1}^N\frac{z^{-1}-\alpha_k}{1-\alpha_kz^{-1}}
\end{equation}
Indsættes
\begin{equation}
H(z)=H_{LP}(Z)\bigg|_{Z^-1=G(z^{-1})}
\end{equation}
For et lavpasfilter
\begin{equation}
G(z)=\frac{z^{-1}-a_k}{1-a_kz^{-1}}
\end{equation}
hvor
\begin{equation}
\alpha=\frac{\sin[(\theta_c-\omega_c)/2]}{\sin[(\theta_c+\omega_c)/2]}
\end{equation}
og
\begin{equation}
\omega=\arctan\left[\frac{(1-\alpha^2)\sin(\theta)}{2\alpha+(1+\alpha^2)\cos(\theta)}
\right]
\end{equation}
\clearpage
\noindent{\Huge 10 Kvantiseringseffekter og skalering af filterstrukturer}\\\\
Ved variabelkvantisering øges SNR med 6dB hver gang opløsningen øges med 1 bit.
\begin{equation}
\text{SNR}_{\text{dB}}=6.02N+1.76\text{dB}
\end{equation}
















\end{document}