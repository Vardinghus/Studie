\documentclass[12pt,a4paper]{report}
\usepackage[utf8]{inputenc}
\usepackage{amsmath}
\usepackage{amsfonts}
\usepackage{amssymb}
\author{Gruppe 7}
\begin{document}
\noindent \textbf{Opgave 1}
For at $\mathbb{C}[z]$ skal være et underrum, skal det opfylde kravene i \textbf{Definition 4.1.1} af vektorrum.\\\\
\textbf{1) Lukkethed under addition}\\
Lad $p,q \in \mathbb{C}[z]$ være polynomier af grad $n$ hhv. med komplekse koefficienter $a_0...a_n$ og $b_0...b_n$. Da gælder, at
\begin{equation}
p+q=(a_0+b_0)+(a_1+b_1)z^1+...+(a_n+b_n)z^n,
\end{equation}
som er et polynomium med komplekse koefficienter $(a_0+b_0)...(a_n+b_n)$. Altså gælder, at $\mathbb{C}[z]$ er lukket under skalarmoltiplikation.\\\\
\textbf{2) Lukkethed under skalarmultiplikation}\\
Lad $p \in \mathbb{C}[z]$ være et polynomium af grad med komplekse koefficienter $a_0...a_n$. Da gælder, at
\begin{equation}
cp=ca_0+a_1z^1+...ca_nz^n
\end{equation}
som er et polynomium med komplekse koefficienter $ca_0...ca_n$. Altså gælder, at $cp \in \mathbb{C}[z]$.\\\\
\textbf{3) Kommutativitet}\\
Lad $p,q \in \mathbb{C}[z]$ være polynomier af grad $n$ hhv. med komplekse koefficienter $a_0...a_n$ og $b_0...b_n$. Da gælder, at
\begin{align}
p+q & =(a_0+b_0)+(a_1+b_1)z^1+...+(a_n+b_n)z^n \\
& = (b_0+a_0)+(b_1+a_1)z^1+...+(b_n+a_n)z^n \\
& = q+p
\end{align}

\textbf{4) Associativitet}\\
Lad $p,q,r \in \mathbb{C}[z]$ være polynomier af grad $n$ hhv. med komplekse koefficienter $a_0...a_n$, $b_0...b_n$ og $c_0...c_n$. Da gælder, at
\begin{align*}
(p+q)+r & =((a_0+b_0)+(a_1+b_1)z^1+...+(a_n+b_n)z^n)+c_0+c_1z^1+...+c_nz^n \\
& =((a_0+b_0+c_0)+(a_1+b_1+c_1)z^1+...+(a_n+b_n+c_n)z^n)\\
& =((a_0+(b_0+c_0))+(a_1+(b_1+c_1))z^1+...+(a_n+(b_n+c_n))z^n\\
& =a_0+a_1z^1+...+a_nz^n+(b_0+c_0)+(b_1+c_1)z^1+...+(b_n+c_n)z^n\\
& = p+(q+r)
\end{align*}
\textbf{5) Additiv identitet}\\
Det trivielle polynomium $p=0$ er den additive identitet.\\\\
\textbf{6) Additiv invers}\\
Lad $p \in \mathbb{C}[z]$ være et polynomium af grad med komplekse koefficienter $a_0...a_n$. Lad $q=-p$. Da gælder, at
\begin{equation}
p+q=(a_0+(-a_0))+(a_1+(-a_1))z^1+...+(a_n+(-a_n))z^n=0,
\end{equation}
hvilket gør $q$ til den additive invers.\\\\
\textbf{7) Multiplikativ identitet}\\
Polynomiet $p=1$ er den multiplikative identitet.\\\\
\textbf{8) Distributivitet}\\
Lad $p,q \in \mathbb{C}[z]$ være polynomier af grad $n$ hhv. med komplekse koefficienter $a_0...a_n$ og $b_0...b_n$ og $a,b \in \mathbb{F}$. Da gælder, at
\begin{align*}
a(p+q) & =a(a_0+b_0)+a(a_1+b_1)z^1+...+a(a_n+b_n)z^n \\
& =(aa_0+ab_0)+(aa_1+ab_1)z^1+...+(aa_n+ab_n)z^n \\
& =aa_0+aa_1z^1+...+aa_nz^n+ab_0+ab_1z^1+...+ab_nz^n \\
& =ap+aq.
\end{align*}
Der gælder desuden også, at
\begin{align*}
(a+b)p & =(a+b)a_0+(a+b)a_1z^1+...+(a+b)a_nz^n \\
& =(aa_0+ba_0)+(aa_1+ba_1)z^1+...+(aa_n+ba_n)z^n \\
& =aa_0+aa_1z^1+...+aa_nz^n+ba_0+ba_1z^1+...+ba_nz^n \\
& = ap+bp
\end{align*}
Siden $\mathbb{C}[z]$ opfylder ovenstående krav, da er det et vektorrum over $\mathbb{C}$.\\\\
$\mathbb{R}[z]$ er ikke et vektorrum over $\mathbb{C}$, da dette ikke er lukket under skalarmultiplikation.\\
Lad $p$ være et polynomium med reelle koefficienter af grad $n$ i $\mathbb{R}[z]$, hvor $z \in \mathbb{C}$.
\begin{equation}
p=a_0+a_1z^1+...+a_nz^n
\end{equation}
Ved skalarmultiplikation med en kompleks konstant $c$ fås
\begin{equation}
cp=ca_0+ca_1z^1+...+ca_nz^n,
\end{equation}
som er et polynomium med komplekse koefficienter $ca_0...ca_n$, og altså gælder, at $cp \notin \mathbb{R}[z]$. Altså er $\mathbb{R}[z]$ over $\mathbb{C}$ ikke et vektorrum.\\\\
For $\mathbb{R}[x]$ over $\mathbb{C}$, hvor $x$ er en reel variabel gælder samme som for $\mathbb{R}[z]$ over $\mathbb{C}$, hvor $z$ er en kompleks variabel - det er ikke et vektorrum.
\end{document}