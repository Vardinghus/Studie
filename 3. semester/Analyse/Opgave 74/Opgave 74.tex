\documentclass[12pt,a4paper]{report}
\usepackage[utf8]{inputenc}
\usepackage{amsthm}
\usepackage{amsmath}
\usepackage{amsfonts}
\usepackage{amssymb}
\author{Gruppe 7}
\title{Opgave 74}
\begin{document}
\noindent \textbf{Opgave 74} Lad delmængden $A \subseteq \mathbb{R}$ have den mindste øvre grænse $b$. Vis, at hvis $b \in A$, så er $b$ det største element i $A$.
\begin{proof}[Bevis]
Ifølge \textbf{Definition 3.10} er $b \in \mathbb{R}$ mindste øvre grænse for en mængde $A \subseteq \mathbb{R}$, hvis
\begin{itemize}
\item[] $ \forall a \in A : b \geq a$.
\item[] $c$ er en øvre grænse for $A$, så er $b \leq c$.
\end{itemize}
Antag, at $b \in A$. Da gælder stadig, at
\begin{itemize}
\item[] $ \forall a \in A : b \geq a$,
\end{itemize}
hvilket ifølge \textbf{Definition 3.2} gør $b$ til det største element i $A$.
\end{proof}

\end{document}