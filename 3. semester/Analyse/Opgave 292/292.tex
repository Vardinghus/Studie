\title{Opgave 164 \\ \small Gruppe 7}
\date{\today}
\documentclass[12pt]{article}

\usepackage{amsmath}
\usepackage{amssymb}
\usepackage{amsthm}
\usepackage[
%  disable, %turn off todonotes
 colorinlistoftodos, %enable a coloured square in the list of todos
 textwidth=\marginparwidth, %set the width of the todonotes
 textsize=scriptsize, %size of the text in the todonotes
 ]{todonotes}

\usepackage[utf8]{inputenc}
\usepackage[danish,english]{babel}
\renewcommand\qedsymbol{$\blacksquare$}

\begin{document}
\maketitle

\paragraph{Opgave 292} - Bevis for L'Hopitals regel om $0/0$-udtryk ved grænseovergange $x \to a^-$ og $x \to a$\\

\subparagraph{a)}  Tilfældet $x \to a^-$ \\
Antag, at $f$ og $g$ er definerede i et interval $\left] a - \rho,a\right[$ til venstre for punktet $a$, og at der gælder
\begin{align}
f(x) \to 0 \ \text{for} \ x \to a^- \label{1}\\
g(x) \to 0 \ \text{for} \ x \to a^- \label{2}\\
\frac{f'(x)}{g'(x)}\to c \ \text{for} \ x \to a^- \label{3}
\end{align}
Bevis, at så gælder der
\begin{equation}
\frac{f(x)}{g(x)} \to c \ \text{for} \ x \to a^-.
\end{equation}
\begin{proof}[Bevis] Lad $\tilde{f}(x)=f(-x)$ og $\tilde{g}(x)=g(-x)$. $\tilde{f}$ og $\tilde{g}$ er da differentiable i intervallet $]-a,-a+\rho[$ og det ønskes bevist, at
\begin{equation}
\frac{\tilde{f}(x)}{\tilde{g}(x)} \to c \ \text{for} \ x \to (-a)^+.
\label{4}
\end{equation}
Det følger af definitionen af $\tilde{f}$ og $\tilde{g}$, at $\tilde{f}'(x)=-f'(-x)$ og $\tilde{g}'(x)=-g'(x)$, hvilket fra \eqref{3} giver
\begin{equation}
\frac{\tilde{f}'(x)}{\tilde{g}'(x)}=\frac{-f'(-x)}{-g'(-x)}=\frac{f'(-x)}{g'(-x)} \to c \ \text{for} \ x \to (-a)^+
\label{5}
\end{equation}
Da antagelsen i \eqref{5} er tilsvarende den for grænseovergangen $x \to a^+$ i Sætning 7.20 er beviset for \eqref{4} analogt med dette, og der fås, at
\begin{equation}
\frac{\tilde{f}(x)}{\tilde{g}(x)}=\frac{f(-x)}{g(-x)}\to c \ \text{for} \ x \to (-a)^+ \ \Leftrightarrow \ \frac{f(x)}{g(x)} \to c \ \text{for} \ x \to a^-.
\end{equation}
\end{proof}
\subparagraph{b)}  Tilfældet $x \to a$ \\
Antag, at $f$ og $g$ er definerede i nærheden af punktet $a$, og at der gælder
\begin{align}
f(x) \to 0 \ \text{for} \ x \to a \\
g(x) \to 0 \ \text{for} \ x \to a \\
\frac{f'(x)}{g'(x)}\to c \ \text{for} \ x \to a
\end{align}
Bevis, at så gælder der
\begin{equation}
\frac{f(x)}{g(x)} \to c \ \text{for} \ x \to a
\label{6}
\end{equation}
\begin{proof}[Bevis] Da grænseværdierne
\begin{align}
\frac{f(x)}{g(x)} \to c \ \text{for} \ x \to a^+ \\
\frac{f(x)}{g(x)} \to c \ \text{for} \ x \to a^-
\end{align}
er bevist hver for sig, følger det af Sætning 6.40, at grænseværdien \eqref{6} eksisterer.
\end{proof}
\end{document}