%  A simple AAU report template.
%  2015-05-08 v. 1.2.0
%  Copyright 2010-2015 by Jesper Kjær Nielsen <jkn@es.aau.dk>
%
%  This is free software: you can redistribute it and/or modify
%  it under the terms of the GNU General Public License as published by
%  the Free Software Foundation, either version 3 of the License, or
%  (at your option) any later version.
%
%  This is distributed in the hope that it will be useful,
%  but WITHOUT ANY WARRANTY; without even the implied warranty of
%  MERCHANTABILITY or FITNESS FOR A PARTICULAR PURPOSE.  See the
%  GNU General Public License for more details.
%
%  You can find the GNU General Public License at <http://www.gnu.org/licenses/>.
%
\input{setup/preamble.tex}% package inclusion and set up of the document
\input{setup/hyphenations.tex}%
\input{setup/macros.tex}% my new macros
\begin{document}
\begin{itemize}
\item[\checkmark] Fuldstændighed af de reelle tal
\item[] Grænseværdier
\item[\checkmark] Bolzano-Weierstrass
\item[\checkmark] Kvotient- og rodkriteriet
\item[] Kontinuitetsbegreber og -egenskaber
\item[\checkmark] Hovedsætninger om kontinuerte funktioner
\item[\checkmark] Åbne og lukkede mængder
\item[\checkmark] Differentiation
\item[\checkmark] Rolles- og Middelværdisætningen
\item[\checkmark] L'Hôpitals regel
\item[\checkmark] Riemann-integralet
\item[\checkmark] Analysens Fundamentalsætning
\end{itemize}
\chapter{Fuldstændighed af de reelle tal}
\subsubsection{Bøger og noter}
\begin{itemize}
\setlength\itemsep{0em}
\item Funktioner af en og flere variable
\item Punktmængdetopologi, metriske rum og fuldstændighed
\end{itemize}
\subsubsection{Tidligere kapitler}
\begin{itemize}
\setlength\itemsep{0em}
\item Korollar 1.9 (Egenskaber for legemer)
\end{itemize}
\subsubsection{Definitioner vi skal have styr på}
\begin{itemize}
\setlength\itemsep{0em}
\item 3.1 (Ordnet mængde)
\item 3.2 (Største/mindste element)
\item 3.6 (Øvre grænse)
\item 3.10 (Mindste øvre grænse)
\item 3.15 (Supremum/infimum)
\item 3.4 (De reelle tal)
\end{itemize}
\subsubsection{Sætninger/lemmaer vi skal have styr på med bevis}
\begin{itemize}
\setlength\itemsep{0em}
\item 3.7 (Ækvivalente udsagn for øvre grænse)
\item 3.8 (Ækvivalente udsagn for !øvre grænse)
\item 3.11 (Mindste øvre grænse)
\item 3.12 (Mindste øvre grænse - anden måde)
\item 3.14 (Infimumegenskaben)
\item 3.17 (Arkimedes' princip)
\end{itemize}
\subsubsection{Sætninger, som kan bevises til eksamen}
\begin{itemize}
\item 4.16 (Hovedsætning om monotone talfølge) (se 3.4 på 255)
\item 4.42 (Enhver Cauchy-følge i $\mathbb{R}$ eller $\mathbb{C}$ er konvergent)
\item 3.4 i (PMT) (Fuldstændighed af $\mathbb{R}$)
\end{itemize}
\clearpage
\begin{theorem}[Fuldstændighed af de reelle tal]Antag, at $K$ er et legeme, som er ordnet, og fuldstændigt og som opfylder, at for ethvert $k \in K$ eksisterer  et $n \in \mathbb{N}$ så $k < n$. Så er $K=\mathbb{R}$.
\end{theorem}
$\mathbb{R}$ er defineret som et ordnet legeme, som opfylder supremumegenskaben. Der vises, at enhver ikke-tom opadtil begærnset delmængde af $K$ har en mindste øvre grænse.
\begin{proof} for Lad $S \in K$ være ikke-tom og opad begrænset. Så eksisterer en øvre grænse $b \in K$ og et $a \in S$, således 
\begin{equation}
\exists -m,M \in\mathbb{N}: m < a \leq b < M
\end{equation}
Definér
\begin{equation}
T_k = \{p \in \mathbb{Z}|p^kM \leq M\text{ og }\frac{p}{2^k}\text{ er en ø.g. for }S\}
\end{equation}
$T_k$ er
\begin{itemize}
\item nedadtil begrænset af $2^km$
\item opadtil begrænset af $2^kM$
\item ikke-tom, da $2^kM \in S$
\end{itemize}
Da $T_k$ er ikke-tom og begrænset, så har den et mindste element (da den er delmængde af $\mathbb{Z}$).\\
Mindste element kaldes $p_k = \min T_k$.\\
Konstruér følge $a_k = \frac{p_k}{2^k}$, som er ø.g. for $S$. Da $p_k$ er mindste elemtn, som opfylder, at $a_k$ er en ø.g. kan $\frac{p_k-1}{2^k}$ ikke være ø.g.\\
Der fås, da de venstre brøker ikke er ø.g., at
\begin{equation}
\frac{p_k-1}{2^k} = \frac{2p_k-2}{2^{k+1}} < a_{k+1} = \frac{p_{k+1}}{2^{k+1}} \leq \frac{2p_k}{2^k} = a_k
\end{equation}
hvilket betyder, at
\begin{equation}
p_{k+1} = 2p_k \text{ eller } p_{k+1} = 2p_k
\end{equation}
Næste skridt er altså mindre-end-lig det foregående. Derfor
\begin{equation}
0 \leq a_k - a_{k+1} = \frac{p_k}{2^k} - \frac{p_{k+1}}{2^{k+1}} \leq \frac{1}{2^{k+1}}
\end{equation}
Lad $m > n \geq 1$. Så er
\begin{align}
0 &\leq a_n - a_m\\
&= (a_n - a_{n+1})+(a_{n+1}-a_{n+2})+\hdots+(a_{m-2}-a_{m-1})+(a_{m-1}-a_m)\\
&\leq \frac{1}{2^{n+1}} + \frac{1}{2^{n+2}} +\hdots+\frac{1}{2^{m-1}}\\
&=\frac{1}{2^{n+1}}\left(\frac{1}{2^0}+\frac{1}{2^1}+\hdots+\frac{1}{2^{m-n-2}}+\frac{1}{2^{m-n-1}}\right)\\
&=\frac{1}{2^{n+1}}\left(\frac{\frac{1}{2^{m-n}}-1}{\frac{1}{2}-1}\right)\\
&=\frac{1}{2^{n+1}}\left(2-\frac{1}{2^{m-n-1}}\right)\\
&=\frac{1}{2^n}-\frac{1}{2^m}\\
&\leq \frac{1}{2^n}
\end{align}
$\{a_k\}_{k=0}^\infty$ er altså Cauchy.\\
Sæt $s=\lim_{n\to\infty}a_k$ og antag $s$ ikke ø.g.
\begin{equation}
\exists x \in S: x > s
\end{equation}
Lad $\varepsilon = x - s > 0$ og find $k$ således
\begin{equation}
x - s = \varepsilon > |a_k - s| = a_k - s\text{ ($a_k$ er aftagende)}
\end{equation}
$x > a_k$ er modstrid, da $a_k = \frac{p_k}{2^k}$, som var ø.g. for $S$.\\
Antag for modstrid, at $s' < s$.\\
Find $k$, så $\frac{1}{2^k} < s-s'$.
$s' < s-\frac{1}{2^k} \leq a_k-\frac{1}{2^k} = \frac{p_k-1}{2^k}$
Ovenstående siger, at $\frac{p_k-1}{2^k}$ er en ø.g., hvilket er i modstrid med tidligere.
\end{proof}
\clearpage
\chapter{Grænseværdier}
\subsubsection{Definitioner der skal være styr på}
\begin{itemize}
\setlength\itemsep{0em}
\item 4.1 (Talfølger)
\item 4.2 (Konvergens af talfølger)
\item 4.6 (Begrænset talfølge)
\item 4.9 (Begrænsethed af konvergent talfølge)
\item 4.17 og 4.20 (Divergens mod $\pm \infty$)
\end{itemize}
\subsubsection{Sætninger, som kan bevises til eksamen}
\begin{itemize}
\setlength\itemsep{0em}
\item 6.41 (Grænseværdi for en monoton funktion)
\item 6.34 eller 6.44 (Grænseværdi for en sammensat funktion. II)
\end{itemize}
\clearpage
\begin{theorem}[Grænseværdi for en sammensat funktion]
Lad $f:]a,\infty[\to\mathbb{R}$ og $g:]b,\infty[\to\mathbb{R}$ være to reelle funktioner således, at $g(x)\in ]a,\infty[$ for $x\in ]b,\infty[$.\\
Bevis, at hvis $g(x)\to\infty$ for $x\to\infty$ og $f(x)\to c$ for $x\to\infty$, så gælder $f(g(x))\to c$ for $x\to\infty$.
\end{theorem}
\begin{proof}
\textbf{(i)} Bevis, at
\begin{equation}
\forall \varepsilon >0\exists\k >0: x \geq k\phantom{mm}\Rightarrow\phantom{mm}|f(g(x))-c|<\varepsilon
\end{equation}
Ifølge definition 6.42
\begin{equation}
\forall \varepsilon >0\exists k\in\mathbb{R}:x\geq k\phantom{mm}\Rightarrow\phantom{mm} |f(x)-c|<\varepsilon
\end{equation}
På samme vis
\begin{equation}
\forall k\in\mathbb{R}\exists \tilde{k}>0:x\geq \tilde{k}\phantom{mm}\Rightarrow\phantom{mm}g(x)>k
\end{equation}
Dermed haves
\begin{equation}
\forall \varepsilon >0\exists \tilde{k}>0:x\geq\tilde{k}\phantom{mm}\Rightarrow\phantom{mm}|f(g(x))-c|<\varepsilon
\end{equation}
\textbf{(ii)} Bevis, når $c=\pm\infty$.\\
Antag UTAG, at $c=+\infty$.\\
Der kan nu skrives (ifølge 6.38)
\begin{equation}
\forall M\in\mathbb{R}\exists K\in\mathbb{R}:x\geq K\phantom{mm}\Rightarrow\phantom{mm}f(x)>M
\end{equation}
og på samme måde for $g$
\begin{equation}
\forall N\in\mathbb{R}\exists \tilde{K}\in\mathbb{R}:x\geq \tilde{K}\phantom{mm}\Rightarrow\phantom{mm}g(x)>N
\end{equation}
Altså haves
\begin{equation}
\forall M\in\mathbb{R}\exists\tilde{K}\in\mathbb{R}:x\geq K\phantom{mm}\Rightarrow\phantom{mm}f(g(x))>M
\end{equation}
Der haves altså, at $f(x)\to\infty$ for $x\to\infty$.
\end{proof}
\clearpage
\chapter{Bolzano-Weierstrass}
\subsubsection{Definitioner}
\begin{itemize}
\setlength\itemsep{0em}
\item 6.9 (Indre punkt)
\item 6.10 (Åben mængde)
\item 6.11 (Lukket mængde)
\item 6.13 (Begrænset mængde)
\end{itemize}
\subsubsection{Sætninger/lemmaer der skal være styr på med bevis}
\begin{itemize}
\setlength\itemsep{0em}
\item 4.16 (Hovedsætning om monotone talfølger)
\item 4.26 (Konvergent/divergent følge og delfølge)
\item 4.27 (Monoton delfølge)
\item 6.5 (Koordinatvis konvergens)
\end{itemize}
\subsubsection{Sætninger, som kan bevises til eksamen}
\begin{itemize}
\setlength\itemsep{0em}
\item Forklar BWI (4.28), bevis BWII (4.29) og forklar BWIII (6.7) om punktfølger
\item Inddrag punktmængdetopologi i forbindelse med BW-egenskab (6.14) - enhver kompakt delmængde af $\mathbb{R}^n$ er følgekompakt
\end{itemize}
\begin{theorem}
Enhver begrænset kompleks talfølge har en konvergent delfølge.
\end{theorem}
\begin{proof}
Lad $\{z_n\}_{n=0}^\infty$ være en begrænset kompleks talfølge.\\
Skriv $z_n=x_n+iy_n$ - altså udtrykt ved reelle talfølger. Ifølge BWI har $\{x_n\}_{n=0}^\infty$ en konvergent delfølge $\{x_{n_k}\}_{k=0}^\infty$. Sæt $\lim_{k\to\infty}x_{n_k}=x$.\\
Følgen $\{y_{n_k}\}_{k=0}^\infty$ har ifølge BWI en konvergent delfølge $\{y_{n_{k_j}}\}_{j=0}^\infty$.\\
Sæt $\lim_{j\to\infty}y_{n_{k_j}}=y$.
Delfølgen $\{x_{n_{k_j}}\}_{j=0}^\infty$ har samme grænseværdi $x$ som $\{x_{n_k}\}_{k=0}^\infty$.\\
Delfølgen $\{z_{n_{k_j}}\}_{j=0}^\infty$ er nu ifølge sætning 4.14 konvergent.
\end{proof}
\clearpage
\chapter{Kvotient- og rodkriteriet}
\subsubsection{Hav styr på}
\begin{itemize}
\setlength\itemsep{0em}
\item 4.35 (Halelemma, indskudsregel for rækker)
\item 4.34 (Sammenligningskriteriet)
\item 4.11 (Konvergent delfølge større end et tal fra et vist trin)
\end{itemize}
\subsubsection{Disposition}
\begin{itemize}
\setlength\itemsep{0em}
\item Forklar talfølge, række, kvotientrække og konvergens af denne og/eller sammenligningskriteriet og halelemmaet
\item Bevis rod- og/eller kvotientkriteriet for rækker i bogen (4.36 og 4.37) eller for potensrækker i noterne (Sætning 6) og fortsæt med konvergensradius
\end{itemize}
\clearpage
\begin{theorem}[Kvotientkriteriet]
Lad $\sum_{n=1}\infty a_n$ være en række med positive led og antag, at grænseværdien
\begin{equation}
q=\lim_{n\to\infty}\frac{a_{n+1}}{a_n}
\end{equation}
eksisterer (den må gerne være $\infty$). Så gælder
\begin{itemize}
\setlength\itemsep{0em}
\item[\textbf{a}] Hvis $q<1$, er rækken konvergent.
\item[\textbf{b}] Hvis $q>1$ er rækken divergent.
\end{itemize}
\end{theorem}
\begin{proof}
\textbf{(a)} Antag $q < 1$ og lad $r>q$. Find ved lemma 4.10 (større end et tal fra et vist skridt) et $N\in\mathbb{N}$ således
\begin{equation}
q < \frac{a_{n+1}}{a_n} < r\text{ for } n \geq N
\end{equation}
Hvilket medfører
\begin{equation}
a_{n+1}<a_nr
\end{equation}
Ved at føre dette videre i successive skridt fra $N$ fås
\begin{align}
&a_{N+1}<a_Nr\\
&a_{N+2}<a_{N+1}r<a_Nr^2\\
\vdots\\
&a_{N+k}<a_Nr^k
\end{align}
Det $k$'te led i halen af den oprindelige række er mindre end det $k$'te led af kvotientrækken $\sum_{n=1}^\infty a_Nr^n$, som er konvergent, da $r<1$. Sammeligningskriteriet giver, at halen er konvergent, og halelemmaet giver, at hele rækken er konvergent.
\textbf{(b)} Antag $q>1$. Fra lemma 4.10 (større end et tal fra et vist skridt) fås et $N\in\mathbb{N}$ således at
\begin{equation}
n\geq N \phantom{mm} \Rightarrow \phantom{mm} \frac{a_{n+1}}{a_n}>1\phantom{mm} \Rightarrow \phantom{mm} a_{n+1}>a_n
\end{equation}
Rækkens led går ikke mod 0, og rækken konvergerer altså ikke.
\end{proof}
\begin{theorem}[Konvergensradius] Hvis grænseværdien 
\begin{equation}
R=\lim_{n\to\infty}\frac{|a_n|}{a_{|n+1|}}
\end{equation}
eksisterer (må gerne være $\infty$), da er $R$ konvergensradius for enhver potensrække på formen $\sum_{n=0}^\infty a_n(z-a)^n$.
\end{theorem}
\begin{proof}
Når $z\neq a$ giver antagelsen om $R=\lim_{n\to\infty}\frac{|a_n|}{|a_{n+1}|}$, at 
\begin{equation}
\lim_{n\to\infty}\frac{|a_{n+1}(z-a)^{n+1}|}{|a_n(z-a)^n|}=\frac{|z-a|}{R}
\end{equation}
som har samme form som ovenstående kvotientkriterium.\\
Altså
\begin{align}
|z-a|<R \phantom{mm} &\Rightarrow \phantom{mm} \frac{|z-a|}{R}<1\\
|z-a|>R \phantom{mm} &\Rightarrow \phantom{mm} \frac{|z-a|}{R}>1
\end{align}
Dermed er $R=\lim_{n\to\infty}\frac{|a_n|}{|a_{n+1}|}$ konvergensradius for rækken.
\end{proof}
\clearpage
\chapter{Kontinuitetsbegreber og -egenskaber}
\subsubsection{Ting der skal være styr på}
\begin{itemize}
\setlength\itemsep{0em}
\item 5.3 (Kontinuitet i et punkt)
\item 5.6 (Følgekontinuitet)
\item 5.7 (Følgekarakterisation af kontinuitet)
\item 5.8 (Kontinuitet (i alle punkter))
\end{itemize}
I kapitel 6 defineres kontinuitet for funktioner af flere variable, og definitioner og sætninger er analoge med dem i kapitel 5. I stedet for absolutværdi bruges normen.\\
Der findes desuden disse yderligere kontinuitetsbegreber i kapitel 6 og 7:
\begin{itemize}
\setlength\itemsep{0em}
\item 6.28 (Uniform kontinuitet)
\item 6.29 (Uniformt kontinuert funktion på kompakt mængde)
\item 7.5 (Hvis $f$ er differentiabel i $a$ er $f$ kontinuert i $a$)
\end{itemize}
For vektorfunktioner gælder samme begreber om kontinuitet, hvis alle funktionerne i vektorfunktionen er kontinuerte. Se 6.45-6.47.
\subsubsection{Disposition}
\begin{itemize}
\setlength\itemsep{0em}
\item Forklar kontinuitet ved topologiske rum
\item Bevis, at $f$ er kontinuert i "$\varepsilon$-$\delta$"-forstand hvis
og kun hvis $f$ er kontinuert mellem de topologiske rum $(\mathrm{A}, \mathrm{T}_\mathrm{A})$ og ($\mathbb{R}^m$, $\mathrm{T}_{\mathbb{R}^m}$)
ifølge Definition 1.3, hvor $\mathrm{T}_\mathrm{A}$ er sportopologien (se Definition 1.4).
\end{itemize}
%\clearpage
%\begin{proof}"$\Rightarrow$"
%Antag, at $f$ er kontinuert i "$\varepsilon$-$\delta$"-forstand. Altså
%\begin{equation}
%\forall \varepsilon >0 \exists \delta >0\forall x\in A:||x-a||<\delta\phantom{mm}\Rightarrow\phantom{mm}||f(x)-f(a)||<\varepsilon
%\end{equation}
%Antag, at $a$ ligger i $D_f$ - altså $a\in f^{-1}(U)$ hvor $U\in \mathcal{T}_{\mathbb{R}^m}$.
%
%\end{proof}
\clearpage
\chapter{Hovedsætninger om kontinuerte funktioner}
\subsubsection{De tre hovedsætninger}
\begin{itemize}
\setlength\itemsep{0em}
\item 5.13 (Begrænsethed)
\item 5.14 (Eksistens af største- og mindsteværdi)
\item 5.16 (Mellemværdisætningen)
\end{itemize}
Sætningerne 6.26 og 6.27 siger tilsvarende om kontinuerte funktioner af flere variable på kompakte mængder.
\clearpage
\begin{theorem}[Mellemværdisætningen]
Hvis funktionen $f:[a,b]\to\mathbb{R}$ er kontinuert og $d$ er et reelt tal, der ligger strengt mellem $f(a)$ og $f(b)$, så finder der et tal $c\in[a,b]$ således, at $f(c)=d$.
\end{theorem}
Til dette skal bruges følgende lemma.
\begin{lemma}
Hvis funktionen $f:[a,b]\to\mathbb{R}$ er kontinuert, og der gælder $f(a)<0, f(b)>0$, så har $f$ et nulpunkt $c\in[a,b]$.
\end{lemma}
\begin{proof}
Betragt $A=\{x\in[a,b]|f(x)\leq0\}$. Da er $A$
\begin{itemize}
\setlength\itemsep{0em}
\item ikke-tom, da $a\in A$
\item opad begrænset, da $x\in A\leq b$
\end{itemize}
Sæt $c=\sup A$ og vis, at $0\leq f(c)\leq 0$.\\\\
\textbf{Bevis for }$\mathbf{f(c)\leq 0}$\\
Opskriv
\begin{equation}
\forall\varepsilon>0\exists x\in A: x>c-\varepsilon\text{ for } \varepsilon=1/n, n=1,2,\hdots
\end{equation}
Da $c$ ø.g. og $\forall n\in\mathbb{N}\exists x_n:$
\begin{equation}
c-\frac{1}{n}<x_n\leq c
\end{equation}
Altså $x_n\to c\text{ for } n\to\infty$.\\
$f$ er kontinuert og dermed følgekontinuert, så
\begin{equation}
f(x_n)\to f(c)\text{ for }n\to\infty
\end{equation}
Da $x_n\in A$ følger at $f(x_n)\leq 0$ og dermed fra korollar 4.12 (om grænseværdier af en følge, som er $\leq 0$) at $f(c)\leq 0$.\\\\
\textbf{Bevis for }$f(c)\geq 0$\\
Da $f(c)\leq 0$ og $f(b)>0$ er $c<b$.\\
Konstruér følge $\{x_n'\}_{n=0}^\infty$, hvor $x_n'=c+1/n$. Fra et vist trin er $x_n'\in[a,b]$ og $x_n'\to c\text{ for }c\to\infty$. Da $c$ er ø.g. for $A$ og $x_n'>c$ er $x_n'\notin A$ og altså $f(x_n')>0$. Altså fås med korollar 4.12, at $f(c)\geq 0$.
\end{proof}
\begin{proof}[Bevis for Middelværdisætningen]
$d$ ligger strengt imellem $f(a)$ og $f(b)$ - altså
\begin{equation}
f(a)<d<f(b)\text{ eller }f(a)>d>f(b)
\end{equation}
Hvis ovenstående lemma anvendes på funktionen $x\mapsto f(x)-d$ eller $x\mapsto d-f(x)$ findes $d$ til at være lig $c$ iovenstående bevis. Grafen forskydes, således $f(a)<0$ eller $f(a)>0$.
\end{proof}
\clearpage
\chapter{Åbne og lukkede mængder}
\subsubsection{Bogen}
\begin{itemize}
\setlength\itemsep{0em}
\item 6.8 (Kugle)
\item 6.9 (Indre punkt)
\item 6.10 (Åben mængde)
\item 6.11 (Lukket mængde)
\item 6.12 (Følgekarakterisation af lukkede mængder)
\end{itemize}
\subsubsection{Afsnit 1.4 om kompakthed i noterne}
Omhandler lukkede mængder i et topologisk rum. Viser, at lukkede og begrænsede mængder er kompakte.
\begin{itemize}
\setlength\itemsep{0em}
\item Definition 1.12 (Lukket mængde)
\item Definition 1.13 (Kompakt mængde)
\item Definition 1.14 (Hvis $K \in X$ er kompakt, er $f(K)$ kompakt)
\item Proposition 1.15 (Hvis $C \in K$ (kompakt), så er $C$ kompakt)
\item Defition 1.16 (Følgekompakthed)
\end{itemize}
\subsubsection{Afsnit 2.1 om metriske rum i noterne}
\begin{itemize}
\setlength\itemsep{0em}
\item Definition 2.1 (Metrisk rum)
\item Definition 2.2 (Åbne og lukkede kugler)
\item Definition 2.3 (Topologien induceret af en metrik)
\item Definition 2.9 (Hvis $K \in X$ er kompakt, som er $K$ lukket og begrænset)
\end{itemize}
\subsubsection{Afsnit 2.3 om normerede vektorrum i noterne}
\begin{itemize}
\setlength\itemsep{0em}
\item Definition 2.16 (Normeret vetkorrum)
\item Sætning 2.19 (Heine-Borels sætning)
\end{itemize}
\subsubsection{Bevis til eksamen}
Der kan gøres flere ting men 6.12 om følgekarakterisation af lukkede mængder er oplagt. Derefter eventuelt 2.9 i noterne.
\begin{itemize}
\setlength\itemsep{0em}
\item Nem(meste)\\
Definér åbne og lukkede mængder ved definition 1.12 i noterne. Definér derefter kompakte mængder ved topologi (1.13 i noterne) og bevis slutteligt 1.15 om lukkede delmængder af kompakte mængder.
\item Mellem(ste)\\
Definér åbne og lukkede mængder og begrænsethed mht. metriske rum og bevis derefter 2.9 om lukket- og begrænsethed af kompakte mængder i metriske rum.
\item Svær(este)\\
Bevis 2.19 (Heine-Borels sætning) eller tilhørende lemmaer om at kompakte mængder i $\mathbb{R}^n$ er lukkede og begrænsede.
\end{itemize}
\clearpage
\begin{theorem}[Følgekarakterisation af lukkede mængder]
En delmængde $F\subset\mathbb{R}$ er lukket, hvis og kun hvis følgende betingelse er opfyldt:\\\\
\centerline{For enhver punktfølge $\{x^k\}_{k=1}^\infty$ i $F$ gælder, at hvis den er konvergent,}\\ \centerline{så er grænsepunktet $lim_k\to\infty x^k=x\in F$.}
\end{theorem}
\begin{proof}
Det skal bevises, at en vilkårlig punktfølge med alle punkter i $F$ for alle $k$ med en eksisterende grænseværdi har denne grænseværdi i $F$.\\
Antag for modstrid, at $x\in F$. Så ligger $x$ i komplementærmængden $\mathbb{R}^n\textbackslash F$.\\
Der konstrueres en kugle med radius $r$ og centrum $x$. Denne ligger i $\mathbb{R}^n\textbackslash F$. Da $\{x^k\}_{k=1}^\infty$ er konvergent mod $x$ gælder der fra et vist trin, at
\begin{equation}
||x^k-x||<r \phantom{mm}\Rightarrow\phantom{mm}x^k\in\mathbb{R}^n\textbackslash F
\end{equation}
hvilket er modstrid.\\\\
Antag for modstrid, at $F$ ikke er lukket. Så er komplementet $\mathbb{R}^n\textbackslash F$ ikke åbent. Det skal negeres, at alle punkter er indre punkter i komplementet:
\begin{align}
\neg(\forall x\in\mathbb{R}^n\textbackslash F\ \exists r>0&: B_r(x)\subseteq\mathbb{R}^n\textbackslash F)\\
\exists x \in \mathbb{R}^n\textbackslash F \ \forall r>0&: B_r(x)\nsubseteq\mathbb{R}^n\textbackslash F
\end{align}
Kan også skrives som at en kugle på randen af komplementet har en fællesmængde med $F$:
\begin{equation}
\exists x \in \mathbb{R}^n\textbackslash F \ \forall r>0: B_r(x)\cap F \neq \emptyset
\end{equation}
Lad $y^k$ opfylde
\begin{equation}
\forall k\in\mathbb{N}:\exists y^k\in B_r(x)\cap F
\end{equation}
Punktfølgen $\{y^k\}_{n=1}^\infty\in F$ konvergerer altså mod $x$, som derfor på være i $F$. Altså er modstrid opnået og $F$ er lukket.
\end{proof}
\begin{theorem}[Kompakt delmængd af metrisk rum er lukket og begrænset]
Lad $(X,d)$ være et metrisk rum. Hvis $K\subset X$ er kompakt, så er $K$ lukket og begrænset.
\end{theorem}
\begin{proof}
Antag for modstrid, at $K$ ikke lukket. Så er komplementet $K^c$ ikke åbent.\\
Der eksisterer et punkt $a\in K^c$ som ikke er et indre punkt. Fællesmængde mellem kugle på randen og $K$ er ikke-tom.
\begin{equation}
\forall n\in\mathbb{N}:\overline{B_{\frac{1}{n}}(a)}\cap K\neq\emptyset
\end{equation}
Den åbne overdækning
\begin{equation}
\left\{\overline{B_{\frac{1}{n}}(a)}^c\right\}_{n=1}^\infty
\end{equation}
kan ikke udtryndes til en endelig overdækning i modstrid med at $K$ er kompakt.\\
Da $K\in X$ er kompakt og $a\in X$ findes en endelig udtynding af $X = \cup_{n=1}^\infty B_n(a)$ således $K\subset\cup_{i\in J}^\infty=B_{\max\{n_i|i\in J\}}(a)$. Altså er $K$ begrænset.
\end{proof}
\clearpage
\chapter{Differentiation}
\subsubsection{Definitioner}
\begin{itemize}
\setlength\itemsep{0em}
\item 5.24 (Arcus sinus)
\item 5.26 (Arcus tanges)
\item 6.9 (Indre punkt)
\item 6.10 (Åben mængde)
\item 7.1 (Differentialkvotient)
\end{itemize}
\subsubsection{Sætninger og lemmaer}
\begin{itemize}
\setlength\itemsep{0em}
\item 5.9 (Sum og produkt af kontinuerte funktioner)
\item 5.11 (Sammensat funktion)
\item 5.23 (Omvendt funktion)
\item 6.35 (Grænseværdi og kontinuitet)
\item 7.2 (Vigtige differentialkvotienter)
\item 7.6 (Kædereglen)
\item 7.7 (Omvendt funktion)
\item 7.8 (Arcus-funktionernes differentialkvotienter)
\end{itemize}
\subsubsection{Vigtige sætninger/definitioner}
\begin{itemize}
\setlength\itemsep{0em}
\item Definition 7.3 ($o$-funktion)
\item Lemma 7.4 (Hvis $f$ er differentiabel findes en $o$-funktion...)
\item Sætning 7.5 (Kontinuert hvis differentiabel) - gør brug af 7.4, 5.9 og 5.11 til bevis
\item Sætning 7.1 (Regneregler for diffferentiation)
\end{itemize}
\clearpage
\begin{theorem}[Differentiation med o-funktion]
Se side 112 i bogen.
\end{theorem}
\begin{proof}
\textbf{(a)}\\
Hvis første ligning i sætningen skal gælde, skal $h\neq 0$ medføre
\begin{equation}
\varphi (h)=\frac{f(a+h)-f(a)}{h}-f'(a)
\end{equation}
Da ovenstående ikke er defineret ved $h=0$ omskrives og $\varphi (h)$ defineres ved 
\begin{equation}
\varphi (h)=\begin{cases}
\frac{f(a+h)-f(a)}{h}-f'(a)&\text{ for } h\neq 0\\
0&\text{ for } h=0
\end{cases}
\end{equation}
Så er det opfyldt, at $\varphi (0)=0$ og $\varphi (h)\to 0\text{ for }h\to 0$.\\\\
\textbf{(ii)}\\
Hvis 
\begin{equation}
f(a+h)=b+\alpha h+\varphi (h)h
\end{equation}
fås
\begin{equation}
\frac{f(a+h)-f(a)}{h}=\alpha+\varphi (h)
\end{equation}
som har grænseværdien $\alpha$ for $h\to 0$.
\end{proof}
\begin{theorem}
Hvis $f$ er differentiabel i $a$, så er $f$ kontinuert i $a$.
\end{theorem}
\begin{proof}
Af lemma 4 følger, at
\begin{equation}
f(x)=f(a)+f'(a)(x-a)+\varphi (x-a)(x-a)
\end{equation}
Hvor højre side er kontinuert ifølge regnereglerne for kontinuerte funktioner (sætning 5.9 og 5.11 om sum af og sammensatte kontinuerte funktioner).
\end{proof}
\clearpage
\chapter{Rolles- og Middelværdisætningen}
En del af beviserne i afsnittet om differentiation bygger på indre punkter og åbne og lukkede kugler og mængder, så hav styr på disse definitioner - også i PMT.
\subsubsection{Sætninger, lemmaer og definitioner}
\begin{itemize}
\setlength\itemsep{0em}
\item Definition 7.14 (Lokalt ekstremum)
\item Sætning 5.14 (Eksistens af største- og mindsteværdi)
\item 7.13 ($f'$ og monotoni)
\end{itemize}
Vær klar på de mange korollarer, som gør brug af $f'$ og $f''$ i forbindelse med lokalt og globalt ekstremum.
\subsubsection{Beviser til eksamen}
Bevis
\begin{itemize}
\setlength\itemsep{0em}
\item 7.9 (Maksimums- og minimumspunkt)
\item 7.10 (Rolles sætning)\\
Her bruges sætning 5.4 og 7.9.
\item 7.11 (Middelværdisætningen)
\end{itemize}
\clearpage
\begin{theorem}[Hældning i ekstremumspunkt er 0]
Lad $I$ være et interval og $c$ et indre punkt i $I$. Hvis $c$ er et ekstremumspunkt for $f$ og $f$ er differentiable i $c$, så er $f'(c)=0$.
\end{theorem}
\begin{proof}
Bevis ved modstrid - hvis $f'(c)\neq 0$, så er $c$ ikke ekstremumspunkt.\\
Da $f$ er differentiabel eksisterer en o-funktion, således
\begin{equation}
f(c+h)=f(c)+(f'(c)+\varphi(h))h\phantom{mm} \forall h\in[-r,r]
\end{equation}
Lad $f'(c)<0$. Ddermed eksisterer et $\delta\in]0,r[$ således $f'(c)+\varphi(h)<0$ for alle $|h|<\delta$ (tjek definition af o-funktion). Dermed følger
\begin{align}
f(c+h)<0&\text{ for }0<h<\delta\\
f(c+h)>0&\text{ for }-\delta<h<0
\end{align}
Altså har $f$ ikke ekstremumspunkt i et punkt $c$, hvor $f'(c)\neq 0$.
\end{proof}
\begin{theorem}[Rolles sætning]
Lad $f:[a,b]\to\mathbb{R}$ være en funktion, som er kontinuert i det lukkede interval $[a,b]$ og differentiabel i det åbne interval $]a,b[$.\\
\centerline{Hvis $f(a)=f(b)=0$, eksisterer det er $c\in]a,b[$ således, at det gælder $f'(c)=0$.}
\end{theorem}
\begin{proof}
Ifølge 5.14 (eksistens af største- og mindsteværdi) har $f$ ekstremumspunkter på intervallet fordi den er kontinuert. I disse ekstremumspunkter er hældningen 0 ifølge ovenstående sætning.\\
Hvis ekstremumspunkterne er i endepunkterne, så er  de både maksimum og minimum og funktionen er dermed konstant.
\end{proof}
\begin{theorem}[Middelværdisætningen]
Lad $f:[a,b]\to\mathbb{R}^n$ være en funktion, som er kontinuert i det lukkede interval og differentiabel i det åbne.\\
\centerline{Så eksisterer der et $c\in]a,b[$ således at}
\begin{equation}
f'(c)=\frac{f(b)-f(a)}{b-a}
\end{equation}
\end{theorem}
\begin{proof}
Lav ret linje mellem endepunkterne:
\begin{equation}
y=\alpha x+b,\phantom{mm}\alpha=\frac{f(b)-f(a)}{b-a}
\end{equation}
Lav kontinuert funktion
\begin{equation}
h(x)=f(x)+(\alpha x+b)
\end{equation}
Se, at $h(a)=h(b)=0$. Da gælder Rolles sætning således der findes et $c$
\begin{equation}
h'(c)=0=f'(c)-\alpha=f'(c)-\frac{f(b)-f(a)}{b-a}
\end{equation}
Dermed er sætningen bevist.
\end{proof}
\clearpage
\chapter{L'Hôpitals regel}
\subsubsection{Definitioner, sætninger og lemmaer der skal være styr på}
\begin{itemize}
\setlength\itemsep{0em}
\item Sætning 4.33 (Konvergent, hvis afsnitsfølgen er begrænset)
\item Sætning 4.34 (Sammenligningskriteriet)\\
Benytter definition 4.33
\item Sætninger og lemmaer om konvergens af Cauchy-følger i $\mathbb{R}$ og $\mathbb{C}$ (4.42-4.45)
\item Sætning 6.35 (Grænseværdi og kontinuitet)
\item Sætning 7.5 (Kontinuert, hvis differentiabel)
\item Sætningerne 7.9, 7.10 og 7.11, som leder op til og er Middelværdisætningen
\end{itemize}
\subsubsection{Sætninger til eksamen}
\begin{itemize}
\setlength\itemsep{0em}
\item Sætning 7.19 (Cauchys middelværdisætning)\\
Benyttes i beviset for 7.20
\item Sætning 7.20 (L'Hôpitals regel om 0/0-udtryk, når $x$ går mod $a$)
\item Sætning 7.21 (L'Hôpitals regel om 0/0-udtryk, når $x$ går mod $\infty$)
\item Sætning 7.22 (L'Hôpitals regel om $\infty / \infty$-udtryk)
\end{itemize}
\clearpage
\begin{theorem}[L'Hôpitals regel om 0/0-udtryk, når $x$ går mod $a$]
Se side 123 i bogen.
\end{theorem}
\begin{proof}
Kun for grænseovergangen $f(x)\to 0$ for $x\to a^+$ for $x\in ]a,a+\rho[$.\\\\
\textbf{(i)}\\
$f$ og $g$ er definerede og kontinuerte i $[a,a+\rho[$, ifølge sætning 6.35 om grænseovergange.\\\\
\textbf{(ii)}\\
Det er antaget, at $f'(x)/g'(x)$ er defineret i et interval $]a,a+\rho_1[$, og derfor $g'(x)\neq 0$.\\
$f$ og $g$ skal være definerede for at de afledede kan, så $\rho_1\leq\rho$.\\
Brøken $f(x)/g(x)$ er også defineret, da $g$ opfylder kravene i Middelværdisætningen og man kan derfor skrive
\begin{equation}
\frac{g(x)-g(a)}{x-a}=g'(\xi)
\end{equation}
og siden $g(a)=0$, så er $g(x)\neq 0$.\\\\
\textbf{(iii)}\\
Bevis
\begin{align}
\forall \varepsilon > 0\exists\delta >0: x-a< \delta\phantom{mm}&\Rightarrow\phantom{mm}\left|\frac{f(x)}{g(x)}-c\right|<\varepsilon\text{ når}\\
\forall \varepsilon > 0\exists\delta >0: x-a< \delta\phantom{mm}&\Rightarrow\phantom{mm}\left|\frac{f'(x)}{g'(x)}-c\right|<\varepsilon
\end{align}
Lad $\varepsilon >0$ være bestemt og $\delta >0$ for nedereste ligning. Dette skal gælde for øverte ligning.\\
$f$ og $g$ opfylder kravene i Cauchys middelværdisætning, så
\begin{equation}
\frac{f(x)}{g(x)}=\frac{f(x)-f(a)}{g(x)-g(a)}=\frac{f(\xi)}{g(\xi)}
\end{equation}
Altså haves
\begin{equation}
\left|\frac{f(x)}{g(x)}-c\right|=\left|\frac{f(\xi)}{g(\xi)}-c\right|<\varepsilon
\end{equation}
Bevis for $x\to a^-$ gøres ved at lave funktioner med omvendt fortegn. Se opgave 292.
\end{proof}
\clearpage
\chapter{Riemann-integralet}
\subsubsection{Definitioner der skal være styr på}
\begin{itemize}
\setlength\itemsep{0em}
\item Definition 8.1 (Inddeling, oversum, undersum)
\end{itemize}
\subsubsection{Beviser og definitioner til eksamen}
\begin{itemize}
\setlength\itemsep{0em}
\item Bevis sætning 8.7 (En funktion er integrabel hvis og kun hvis der findes en fin nok inddeling for $\varepsilon>0$)
\item Bevis sætning 8.9 (En kontinuert funktion er integrabel)\\
Der gøres brug af sætning 6.29 om uniform kontinuitet.
\end{itemize}
\clearpage
\begin{theorem}[Integrabel hviss over- og undersum går mod hinanden]
En begrænset funktion $f$ på et lukket interval er integrabel hvis og kun hvis der for ethvert $\varepsilon >0$ findes en inddeling $D$ af intervallet således
\begin{equation}
O(D)-U(D)=\sum_{i=1}^n(G_i-g_i)(x_i-x_{i-1})<\varepsilon
\end{equation}
\end{theorem}
\begin{proof}
Antag for modstrid at $f$ ikke er integrabel
\begin{equation}
\underline{\int_a^b}\! f(x) \ dx\neq\overline{\int_a^b}\! f(x) \ dx
\end{equation}
Sætning 8.4 giver
\begin{equation}
\underline{\int_a^b}\! f(x) \ dx\leq\overline{\int_a^b}\! f(x) \ dx
\end{equation}
Det vides, at 
\begin{equation}
U(D)\leq\underline{\int_a^b}\! f(x) \ dx\text{ og }O(D)\leq\overline{\int_a^b}\! f(x) \ dx
\end{equation}
Sæt
\begin{equation}
\varepsilon = \overline{\int_a^b}\! f(x) \ dx-\underline{\int_a^b}\! f(x) \ dx
\end{equation}
Omskriv
\begin{equation}
O(D)-U(D)\leq\overline{\int_a^b}\! f(x) \ dx-\underline{\int_a^b}\! f(x) \ dx=\varepsilon
\end{equation}
Summerne konvergerer dermed mod hinanden og er i modstrid med antagelsen. Der er brugt, at $f$ begrænset på et lukket og begrænset interval.\\\\
\textbf{(ii)}\\
Bevis, at summerne går mod hinanden
\begin{equation}
O(D)-U(D)<\varepsilon
\end{equation}
Da integralet er infimum af mængden af oversummer ved forskellige inddelinger fås
\begin{equation}
O(D_1)<\overline{\int_a^b}\! f(x) \ dx + \frac{\varepsilon}{2}
\end{equation}
På samme måde med supremum for undersummerne
\begin{equation}
U(D_2)>\underline{\int_a^b}\! f(x) \ dx - \frac{\varepsilon}{2}
\end{equation}
Det følger dermed, at 
\begin{equation}
O(D)-U(D)<\varepsilon
\end{equation}
$f$ er integrabel hvis og kun ovenstående gælder.
\end{proof}
\clearpage
\chapter{Analysen Fundamentalsætning}
Der skal forberedes på alt fra Analysens Fundamentalsætning til slutningen af kapitel 8.
\subsubsection{Beviser til eksamen}
\begin{itemize}
\setlength\itemsep{0em}
\item Præsentér sætning 8.13 (Indskudsreglen) og 8.14 (Middelværdisætningen for integraler)
\item Bevis Analysens Fundamentalsætning
\end{itemize}
\begin{theorem}[Analysen Fundamentalsætning]
Lad $I$ være et interval og lad $f:I\to\mathbb{R}$ være en kontinuert funktion. Lad endvidere $a\in I$ og lad funktionen $F:I\to\mathbb{R}$ være defineret ved
\begin{equation}
F(x)=\int_a^x \! f(t) \ dt
\end{equation}
Så er $F$ kontinuert på $I$ og differentiabel i det indre af $I$ mid differentialkvotienten
\begin{equation}
F'(x)=f(x)
\end{equation}
\end{theorem}
\begin{proof}
\textbf{(i)}\\
Lad $x$ og $y\in I$. Med indskudsreglen og Middelværdisætningen fås
\begin{equation}
F(y)-F(x)=\int_a^y \! f(t) \ dt-\int_a^x \! f(t) \ dt=\int_x^y \! f(t) \ dt=f(c)(y-a)
\end{equation}
hvor $c$ ligger mellem $x$ og $y$.\\\\
\textbf{(ii)}\\
Bevis, at $F$ kontinuert i $x\in I$. $f$ er kontinuert i $x$, så der findes $r>0$, således
\begin{equation}
|t-x|<r\phantom{mm}\Rightarrow\phantom{mm}|f(t)|\leq K
\end{equation}
Lad vilkårligt $y$ opfylde ovenstående, og et $c$ imellem $y$ og $x$ opfylde $F(y)-F(x)=f(c)(y-x)$. Da $|f(c)|\leq K$ fås $|F(y)-F(x)|\leq K|y-x|$. Hvis $|y-x|$ mindskes er det mindre end et $\varepsilon$ og $F$ er altså kontinuert i $x$.\\\\
\textbf{(iii)}\\
Bevis, at $F$ er differentiabel i $x$ med differentialkvotient $f(x)$.
\begin{equation}
\forall \varepsilon >0\exists\delta >0:0<|y-x|<\delta\phantom{mm}\Rightarrow\phantom{mm}\left|\frac{F(y)-F(x)}{y-x}-f(x)\right|<\varepsilon
\end{equation}
Lad $\varepsilon >0$ være givet. Dermed
\begin{align}
\exists \delta_1 >0\forall t\in I:|t-x|<\delta_1\phantom{mm}&\Rightarrow\phantom{mm}|f(t)-f(x)|<\varepsilon\\
\exists \rho >0\forall t\in\mathbb{R}:|t-x|<\rho\phantom{mm}&\Rightarrow\phantom{mm}t\in I
\end{align}
Sæt $\delta=\min\{\delta_1,\rho\}$ og find $y$ således $|y-x|<\delta<\delta_1\leq\rho$. Et tal $c$ mellem $x$ og $y$ opfylder også ovenstående og ligning i \textbf{(i)}. Dermed
\begin{equation}
\left|\frac{F(y)-F(x)}{y-x}-f(x)\right|=|f(c)-f(x)|<\varepsilon
\end{equation}
Dermed er $F$ differentiabel i $x\in I$ med differentialkvotient $f(x)$.
\end{proof}
\pagestyle{empty} %disable headers and footers
\pagenumbering{roman} %use roman page numbering in the frontmatter
%\lstset{
%	language = C,
%	backgroundcolor=\color{gray},
%	basicstyle=\ttfamily,
%	keywordstyle=\color{red},
%	numberstyle=\tiny\color{black},
%	numbers = left,
%	numbersep=5pt,
%	captionpos=b
%	}
%\def\inline{\lstinline[basicstyle=\ttfamily,keywordstyle={}]}
\pagestyle{fancy} %enable headers and footers again
\pagenumbering{arabic} %use arabic page numbering in the mainmatter
\end{document}