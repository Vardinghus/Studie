\subsection{PCA på billeder}
\begin{document}
dette afsnit omhandler hvordan man bruge PCA (principle component analysis) på et digitalt billede.
for at kunne anvende PCA på et billede skal man først bruge billede matricen, et billede kan ses som et 2 dimensionelt funktion af $f(x,y)$ hvor $x$ og $y$ er koordianter i rummet.for alle værdierne i billedet matricen skal man sætte dem efter hinanden så de danne en række og derefter transponer den så det bliver til en vektor. 
først starter vi med at opstille vores billede matrice hvor, for hver farve i billede (rød,blå,grøn) opstillers som vektorer. 
\begin{align*}
A = 
\begin{bmatrix}
v_[rød1] & v_[blå1] & v_[grøn1] \\
\vdots & \vdots & \vdots \\
v_[rødn] & v_[blån] & v_[grøn n] \\
\end{bmatrix}
\end{align*}
for at kunne anvende PCA på den skal man først finde finde dens gemmensnits vektor, dette gøres ved at tag gemmensnitet for hver kolonne for at danne en ny vektor
\begin{align*}
\overline{\underline{m}}_x = \frac{1}{q}
\begin{bmatrix}
\sum^Q_k=1 x_{1K} \\
\sum^Q_k=1 x_{1K} \\ 
\vdots \\
\sum^Q_k=1 x_{MK}  \\
\end{bmatrix}
=
\begin{bmatrix}
m_1 \\
m_2 \\
\vdots \\
m_M
\end{bmatrix}
\end{align*}
$\tilde{M_x}$ kan så beskrives som, værdierne fra $\overline{\underline{m}}_x$ samt $M$ tid og kan skrives som
\begin{align*}
\tilde{M}_x = [\overline{\underline{m}}_x,\overline{\underline{m}}_x,\cdots,\overline{\underline{m}}_x]
\end{align*} 
\end{document}