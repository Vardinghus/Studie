\chapter{Projektbeskrivelse}
Følgende afsnit har til formål at skabe rammerne for rapportens retning igennem en problemanalyse, problemformulering og et metodeafsnit. Problemanalysen analyserer det initierende problem og undersøger præmisserne for problemet. Dette bliver efterfølgende brugt til klarlægning af problemformuleringen, der har hovedfokus på billedkomprimeringsmetoder vha. lineær algebra.

\section{Problemanalyse}
Digitale billeder er en stor del af mange menneskers hverdag, både direkte og indirekte, og har ligeledes en afgørende betydning for mange store teknologivirksomheders eksistens. Qua den rivende hurtige teknologiudvikling det seneste årti, er kameraer i mobiltelefoner blevet udviklet til at have bl.a. højere opløsning. Højere opløsning betyder, at billedfilerne også fylder betydeligt mere digital hukommelse, hvorved en mobiltelefons lagerplads hurtigere end tidligere opfyldes. Det betyder ydermere, at download og upload til internettet bliver langsommere, da en større mængde data tager længere tid at downloade end en lille mængde data.

Den enkelte bruger, der typisk besidder et stort fotoalbum på sin mobil, påvirkes af den stigende størrelse på billedfilerne. Dette gør de ved, at deres mobiltelefons lagerplads hurtigere bliver opfyldt af billederne, og da de fleste ønsker at gemme så mange billeder som muligt, kan det være interessant at komprimere billederne for at få plads til flere billeder.

Hele essensen i komprimering af billeder er at gøre filerne mindre, hvorved de fylder mindre lagerplads, hvilket kan anses som en billedbehandling. Billedet, der behandles, kan anses som værende en $m \times n$ matrix, hvor $m$ og $n$ er billedets dimensioner i pixels. Indholdet i matricen er billedets pixelværdier, og indeholder forskellige data afhængigt af billedformatet. Billedbehandlinger, såsom komprimering, fremkommer ved operationer på matricen, og er af den matematiske gren lineær algebra, der netop omhandler matrixoperationer.

Der findes mange forskellige komprimeringsmetoder, og disse kan grundlæggende opdeles i to forskellige typer; tabsfri og ikke-tabsfri komprimering. Tabsfri komprimering er en komprimering af dataene uden at fjerne noget data, og ikke-tabsfri er det modsatte; en komprimering med tab af data.\\
JPEG-komprimering er den mest udbredte komprimeringsmetode, og har oprindelse i gruppen med samme initialer; Joint Photographic Experts Group. Metoden som bruges i JPEG er tilnærmelsesvist en \emph{Diskret Cosinus Transformation} (fremover kaldet DCT), som fungerer på baggrund af tesen om at menneskets syn ikke særlig godt ser høje frekvenser, hvorfor disse kan fjernes fra billedet uden betydelig synlig ændring. DCT er en ikke-tabsfri komprimeringsmetode og ved dekodning af den komprimerede fil, vil det genskabte billede kun være en tilnærmelse af det originale, og ikke en eksakt kopi. Forskellen mellem det genskabte og originale billede er de høje frekvenser, og tabet ved komprimeringen burde derfor ikke være synlig for det blotte øje. DCT er datauafhængig og transformationerne fungerer derved på alle billeder (dog med forskellig kvalitet).
En anden metode til komprimering er \emph{Principal Component Analysis} (fremover kaldet PCA), der ligesom DCT er en ikke-tabsfri komprimeringsmetode. PCA er en statistisk metode, der i store mængder data identificerer signifikante (og dermed også insignifikante) data. PCA finder kovariansen mellem dataene og tildeler de enkelte dimensioner en egenvektor med dertilhørende egenværdi, der bruges til vurdering af signifikante og insignifikante data. Akserne med størst varians udtrykker mest af billedet, og akserne med mindst varians fjernes derfor i komprimeringen. Modsat DCT fjernes de høje frekvenser ikke med PCA, men derimod de data der har mindst betydning for billedets repræsentation. Det betyder dog også at billedet (ligesom DCT) blot skaber en tilnærmelse af det oprindelige billede ved dekodning. PCA er dataafhængig, og billedet har større betydning for komprimeringen med PCA end med DCT.

På trods af mange ligheder mellem PCA og DCT, fungerer de i praksis meget forskelligt, hvilket også betyder at de har hver deres fordele og ulemper. Det betyder også at hver metode fungerer bedre på nogle former for billeder end andre, hvorved det kan være svært at finde den bedst egnede metode til et vilkårligt billede.

\subsection*{Hastighed}
I forbindelse med brug af et komprimeringsprogram i dagligdagen er det afgørende, at tiden det tager at udføre komprimeringen, er så kort som mulig, da det må antages at de fleste ønsker, at komprimeringen skal foregå hurtigst muligt. Da projektet ikke har sin kerne i programmering og kvaliteten af de programmer, der er udviklet til komprimeringen, er fokus ikke lagt på dette. Det betyder imidlertid, at hastigheden ikke kan måles i programmerne, der er udviklet til dette projekt. Til gengæld kunne en kompleksitetsanalyse laves, som giver et udtryk for hvor mange beregninger, der skal laves for at lave de to komprimeringsmetoder. Kompleksitetsanalyse ser bort fra den respektive enhed, der laver beregningerne, og giver derfor et udtryk for antal beregninger, der skal laves. Kompleksitetsanalyse er dog også udenfor denne rapports fokusområde og udføres derfor ikke. Bemærk at hastigheden ved komprimering har stor betyding for metodens relevans i virkeligheden men undersøges ikke i denne rapport.

\section*{Case - billeder på mobiltelefoner}
Den følgende rapport har udgangspunkt i den stigende brug af mobiltelefoner som kamera i dagligdagssituationer, hvorved mængden af billeder på den gængse mobiltelefon stiger. Dette giver dog problemer med hurtigt fyldt lagerplads på mobiltelefonerne, og deraf frustrerede brugere. Ved at komprimere billederne opnås der mere fri hukommelse og dermed plads til flere billeder. Der er dog begrænsninger i forhold til kvaliteten af komprimeringen, da den hurtige teknologiudvikling også resulterer i større krav til billederne, der tages. Det er for brugeren vigtigt, at komprimeringen kan være en dagligdagsopgave, og dermed ikke kræver lang tid at udføre. Det er også vigtigt, at billederne bibeholder en vis skarphed og sin farve, for at brugeroplevelsen bibeholdes.

\section{Problemformulering}
I kommende rapport bliver digitale billeder komprimeret vha. lineær algebra og undersøges ud fra problemformuleringen: \\
\begin{itemize}
\item[] \textit{Hvilken af komprimeringsmetoderne DCT og PCA komprimerer bedst billedet Lena \citep{lena}, og hvordan fungerer disse komprimeringsmetoder rent matematisk?} \\
\item[] \textit{Hvilken af metoderne bibeholder bedst kvaliteten af billedet i forhold til komprimeringsgraden?} \\
\item[] \textit{Hvilken metode komprimerer bedst rigtige mobilbilleder, og er der nogen sammenhæng mellem disse resultater?}
\end{itemize}

\section*{Metode}
På trods af at der findes et utal af komprimeringsmetoder, undersøges der i følgende rapport blot komprimeringsmetoderne DCT og PCA. Begge metoder fungerer ved at fjerne de visuelt mindst relevante data om billedet. Komprimeringsmetodernes matematiske grundlag undersøges og bruges efterfølgende til udvikling af python-programmer til komprimering af et billede. Billedet der komprimeres er Lena \citep{lena}, som er $512 \times 512$ pixels stort og i farveformatet RGB (Rød, Grøn, Blå). Resultaterne af komprimeringerne bliver sammenlignet i forhold til parametrene komprimeringsgrad og billedekvalitet. Komprimeringsgraden bliver sammenlignet på baggrund af antal bits i det oprindelige billede i forhold til det komprimerede billede. Billedkvaliteten vurderes på baggrund af forskellene i pixelværdierne mellem det oprindelige og komprimerede billede såvel som ved SNR-værdien for de forskellige komprimeringsgrader. Afslutningsvist bruges komprimeringsprogrammerne på flere forskellige mobilbilleder, for at afklare hvilke parametre, der har betydning for hvilken af metoderne, der fungerer bedst ved dagligdagsbilleder taget med en mobil.

På baggrund af de ovenstående afsnit og dertilhørende data bliver der i afsnit \ref{sec:vurdering} vurderet på kvaliteten og komprimeringsgraden for de to metoder, og sætte dem i relation til casen for rapporten.